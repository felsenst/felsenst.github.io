\documentstyle[12pt]{article}
\def\baselinestretch{1.5}
\setlength{\topmargin}{0pt}
\setlength{\textheight}{570pt}
\setlength{\oddsidemargin}{0pt}
\setlength{\evensidemargin}{60pt}
\setlength{\textwidth}{427pt}
\setlength{\footheight}{0pt}
\setlength{\footskip}{30pt}
\parindent 25pt
\hyphenpenalty=10000
\tolerance=10000
\pagestyle{plain}

\begin{document}
\begin{flushleft}

Applications of Metropolis-Hastings genealogy sampling

\bigskip
\bigskip
Mary K. Kuhner, Jon Yamato, and Joseph Felsenstein

Department of Genetics, University of Washington

\bigskip
Running head:  Metropolis-Hastings sampling

\bigskip
Corresponding author:  
\bigskip

Mary K. Kuhner

Department of Genetics SK-50

University of Washington

Seattle, WA 98195, USA

Phone (206) 543-8751

FAX (206) 543-0754

Internet 
\begin{it}
mkkuhner@genetics.washington.edu
\end{it}
\end{flushleft}

\newpage

{\center Abstract}
\bigskip

The genealogy of a random sample of sequences from a population provides
information on the parameter $\Theta$ (neutral mutation rate times effective
population size).  When the genealogy is unknown, $\Theta$ could in
principle be
estimated by summing over possible genealogies, but there are too
many of them.  We use a
Metropolis-Hastings sampler to concentrate attention on the genealogies
of high posterior probability for a given value of $\Theta$; a
likelihood curve for nearby values of $\Theta$ can be constructed from
the sampled genealogies,
and its maximum is a maximum likelihood estimate of $\Theta$.  
This method can in principle be extended to
populations with migration and population growth, and to sequences with
recombination and selection.  It can also be applied to types of data
other than nucleotide sequences in cases where an appropriate 
likelihood model is 
available.

\bigskip
{\center Introduction}
\bigskip

The relationships among individuals from a population can be summarized
by a genealogy showing the common ancestors of the individuals and the
times at which those ancestors lived.  If the individuals were sampled
at random from a randomly mating population, the times back to the
common ancestors have an approximate expected distribution called the {\it
coalescent} which was derived by Kingman (1982a, b).  The coalescent
distribution depends only on the number of individuals sampled and the
parameter $4N_e\mu$, also called $\Theta$, which combines the effects of 
effective
population size $N_e$ and the neutral mutation rate $\mu$.  (We use
capital $\Theta$ rather than the conventional lowercase $\theta$ in this
paper because we are considering mutation rate per site, not per locus
as in most previous studies.)  Using the coalescent approximation, 
$\Theta$ can be estimated from the genealogy.
Such an estimate is potentially more efficient than one made
from the sequence data without reference to the genealogy (Felsenstein
1992b).
Unfortunately the genealogy is generally unknown.  

At least three approaches have been previously proposed for estimates
of $\Theta$ incorporating genealogical information.
The method of Fu (1994) makes an estimate based on a single 
reconstructed genealogy, combined with a correction factor. 
The method of Tavar\'{e} and Griffiths (1993a, b) sums over all (in
smaller cases) or a random sample of possible genealogical explanations 
for the data.
The bootstrap Monte Carlo method of Felsenstein (1992a) makes an
estimate based on genealogies produced by bootstrapping the original data set.
(We have recently found the bootstrap Monte Carlo method to be biassed
(Kuhner, Yamato and Felsenstein 1995, submitted) and do not recommend its use.)

The approach proposed in the current paper is importance sampling:  we
sample among all
genealogies, but concentrate on those which are expected to make
substantial contributions to the estimate.
This can be done by means of a Metropolis-Hastings sampler.  New
genealogies are created by small modifications of an initial one, and
accepted or rejected based on their fit to the data.  Samples
from this Markov Chain of genealogies can be used to construct a
likelihood curve for $\Theta$ and to find its maximum.  This approach
avoids any bias which may be caused by using a single reconstruction of
the genealogy, since such reconstructions unavoidably involve some
degree of error.

We have designed such a sampler, which can be used for non-recombining
nucleotide sequences (such as mitochondrial DNA) from a constant-size, randomly
mating population.  The details of the sampling procedure are presented
in another paper (Kuhner, Yamato, and Felsenstein 1995, submitted):  this paper
will briefly review the sampler, and discuss ways in which it can be
extended to other types of data.

\bigskip
{\center Methods}
\bigskip

The Metropolis-Hastings sampler on genealogies is a form of importance
sampling:  we sample genealogies from a known distribution which we hope
will be similar to the unknown true distribution in $\Theta$, in order
to gain as much information about $\Theta$ per sampled genealogy as
possible.  In evaluating the genealogies, we then correct for the
known distribution from which we sampled.

The known distribution we will use incorporates the probability of the
sequence data given the genealogy $(P(D|G))$, and the probability of the
genealogy given a chosen value of $\Theta$ which we will call
$\Theta_0$ $(P(G|\Theta_0))$.  The true distribution which we are trying
to estimate is identical to this except that the unknown true $\Theta$
replaces $\Theta_0$.  The importance sampling function is:
\bigskip

\begin{displaymath}
\frac{P(D|G)P(G|\Theta_0)}{P(D|\Theta_0)}
\end{displaymath}

\bigskip

which is the posterior probability of the genealogy.  Since only ratios
of posterior probabilities are needed, the unknown constant
$P(D|\Theta_0)$ need not be considered.

The prior probability, $P(G|\Theta_0))$, can be readily computed using the 
coalescent
approximation of Kingman (1983a, b).
The probability with respect to the data, $P(D|G)$, is the quantity maximized by ML methods of phylogeny
estimation, and can be computed for a variety of models:  we chose the
nucleotide sequence model of Felsenstein (described by Kishino and
Hasegawa 1989) which allows the specification of the ratio of
transitions to transversions and of the frequencies of the four bases.
Under Extensions we describe alternative models which could be used.

To carry out this sampling, we construct a new genealogy by modifiying the
current genealogy, drawing the new coalescence times from a
distribution proportional to $P(G|\Theta_0)$.  (This procedure by itself, if
repeated many times, would eventually sample all genealogies in
proportion to how well they fit coalescent expectations for the given 
$\Theta_0$.)  Having
constructed a new genealogy, we decide whether to accept or reject it by
comparing the relative probability of the sequence data, $P(D|G)$, on
the old and new genealogies.  The outcome of this process, repeated many
times, is a Markov Chain of genealogies where each possible genealogy is
represented in proportion to the product $P(G|\Theta_0)P(D|G)$, its
posterior probability. 
Details of the 
rearrangment procedure follow.

To make a local rearrangement, we choose a random node in the tree and
consider a neighborhood around it:  its two descendent nodes, its parent,
and its parent's other descendent.  A series of neighborhood
rearrangements of this type can eventually transform any topology into
any other.

The interior of the neighborhood is erased and replaced with a new
set of coalescences chosen based on a conditional coalescent distribution.
Two factors must be considered:  (1) the probability of coalescence at a
given moment depends on the number of lineages present in the entire genealogy
at that moment; (2) the three lineages within the neighborhood must
coalesce with one another, and not with any other lineage, by the end of
the neighborhood (this condition limits the process to local
rearrangements). 
We use a modification of the state-array approach of
Viterbi (1967).  Briefly, we construct a lattice containing the
conditional probabilities that three, two or one lineages are present in
the neighborhood at each moment, then trace a weighted random path
through this lattice beginning with one lineage at the bottom of the 
neighborhood and ending with three lineages at the top.  This path
defines a set of coalescence times:  the lineages coalescing are then
chosen at random from the neighborhood lineages available at the given times.
(Complete details of this algorithm are given in Kuhner, Yamato, and
Felsenstein 1995, submitted.)

Now that a modified genealogy has been constructed, the probability of the
data on the old and new genealogies is evaluated under the chosen model.
The decision to accept or reject the new genealogy depends on
the ratio $r$ of these probabilities; if $r\geq 1$ the new genealogy is
accepted with probability 1, whereas if $r<1$ it is accepted with
probability $r$ (Metropolis et al. 1953).  The method of lineage erasure
and redrawing does not introduce any bias towards particular topologies,
so the corrective terms described by Hastings (1970) need not be
computed in this form of the sampler.  However, more complex future 
extensions of the
algorithm may require corrective terms to compensate for any tendency to
over-propose certain genealogies.

At intervals, genealogies can be sampled from this Markov Chain, and the
sampled genealogies can be used to construct a likelihood
curve for $\Theta$ via the following relationship, which compensates for
the use of the importance sampling function:

\begin{equation}
L(\Theta)=\sum_G \frac{P(G|\Theta)}{P(G|\Theta_0)}
\end{equation}

The Metropolis-Hastings sampler should produce the correct value of
$L(\Theta)$
for any value of $\Theta_0$, asymptotically as the number of
steps along the Markov chain approaches infinity.  In practice, for
finite numbers of steps there is a bias towards $\Theta_0$ if it is too
distant from $\Theta$, since genealogies informative for the true value
of $\Theta$ will very rarely be produced.  A
useful approach is to run several chains, each using
as its $\Theta_0$ the maximum of the curve for the previous chain.
Genealogies from all chains can be used to make a single estimate by treating
them as if they were sampled from a mixture distribution (Geyer 1991).

\bigskip
{\center Applications}

The Metropolis-Hastings genealogy sampler was originally designed in order to
estimate $\Theta$, but the ability to sample genealogies in proportion
to their posterior probability has other potential uses.

{\bf Phylogeny estimation.}  The maximum likelihood method of phylogeny 
estimation searches for the phylogeny maximizing $P(D|G)$.
Maximum likelihood
is one of the most effective phylogeny methods (Kuhner
and Felsenstein 1994), but it is computationally intensive.
Current heuristics such as the one implemented in Felsenstein's DNAML
program from PHYLIP are not guarenteed to find the maximum likelihood
phylogeny, and they are very slow on large data sets.

The Metropolis-Hastings sampler is not attempting to maximize $P(D|G)$.
However, in the course of its search
the sampler can easily make note of the highest value encountered, and
the genealogy producing it.
In practice, this genealogy often has the same topology as
the maximum likelihood genealogy.  Its branch lengths can then be 
optimized by standard methods.  Thus, the Metropolis-Hastings sampler
can be used as a heuristic method for finding the genealogy of maximum
likelihod with respect to the data.  It is much faster on large data
sets (50 sequences and up) than DNAML. 
We plan to compare their performance on simulated data.
Gary Olsen (pers. comm.) reports
that DNAML's heuristic frequently becomes trapped in local maxima when
analyzing large data sets.  Preliminary results suggest that the
sampler is less prone to become trapped.

For data drawn from a randomly mating population (as
opposed to data representing multiple populations or species) the
genealogy of maximum posterior probability at the best value of
$\Theta$ may in fact be a superior reconstruction of the true genealogy,
since the coalescent
distribution contributes some information about expected branch
lengths.  However, this approach is probably not
appropriate for interspecific data, since speciation may not follow
coalescent dynamics.

{\bf Bootstrap-like resampling.}  The genealogies sampled during a 
Metropolis-Hastings run with the best
value of $\Theta$ represent a weighted random sample from the space of
all genealogies, and can be used to test hypotheses about that space.
For example, they can be used in the same way as a set of bootstrap
reconstructions to test the strength of support for a particular branch
or rooting.  However, since successive steps of the Markov Chain are not
independent, care must be taken to sample at sufficiently long intervals
that successive sampled genealogies are close to independence.  Even
with a very long sampling interval, the Metropolis-Hastings sampler is
much faster than extant likelihood bootstrapping approaches for medium
to large numbers of sequences.  It should be feasible to apply it to
questions such as the degree of support for an African root in the
human mitochondrial DNA data of Vigilant et al. (1991).

\bigskip
{\center Extensions}
\bigskip

{\bf Other likelihood models.}  Any likelihood model can be substituted into 
this algorithm, leaving the
genealogy-rearrangement machinery intact.  For example, the sequence
likelihood model of Yang (1993), which assumes that mutation rates
follow a gamma distribution, can be used if an efficient method of
computing it is developed.
Types of data other than
nucleotide sequences can be accomodated:  for example, 
the protein likelihood model of 
Kishino et al. (1990),  
or the restriction site likelihood model of Felsenstein (1992c).
These models will, however, be more 
computationally intensive than the current one. 

{\bf Variable population size.}
Change in population size changes the expected shape of the coalescent,
and thus reconstructions of the genealogy can be used to explore past
population size changes.  We are currently exploring estimation of an
exponential growth rate $g$.  The approach is to assume an initial
estimate of $g$ (called $g_0$) and an initial estimate of the present-day
$\Theta$ (called $\Theta_0$).  Modification of
genealogies is then done in much the same way as the original algorithm,
except that times are rescaled in terms of $g_0$.  From the collection
of sampled genealogies it is possible to make a joint estimate of $g$
and $\Theta$, though there will be bias unless $g_0$ and $\Theta_0$ are
fairly close to $g$ and $\Theta$. 

Preliminary results indicate that this method produces an upward bias in
estimation of both $\Theta$ and $g$ (i.e. the population is estimated to
be bigger than it really is, and to have grown more rapidly), but that
use of multiple unlinked loci can decrease this bias.

A further possibility in this direction is to allow several different
episodes of exponential growth or decline, and attempt to estimate the
entire collection of growth rates and time periods.  This is complicated
by the fact that the model with the largest number of episodes will 
tend to fit the data best unless the model explicitly penalizes
rate changes.  It is difficult to suggest a
likelihood model incorporating changes in growth rate, since little
is known about the statistical distribution of such changes.  An
alternative approach would be to run independent Markov Chains with
different numbers of episodes, and then accept the results of the first
chain for which adding a given number of additional episodes does not
significantly improve the results.  

{\bf Recombination.}
The coalescent distribution is very stochastic, and a single realization
of it (such as is provided by mitochrondrial DNA) cannot provide much
certainty about the history of the population.  Multiple independent
samples are desirable, but in order to analyze nuclear DNA we will need to
deal with recombination.

Recombinant genealogies can be thought of as collections of ordinary
genealogies, each applying to a part of the sequence (Figure 1).  There
is no difficulty in evaluating the likelihood of sequence data on such a
genealogy, since each individual site ``sees" a normal genealogy.  The
major issue that must be resolved is how to make rearrangements 
that sample among recombinant as well as non-recombinant
genealogies in proportion to $P(G|\Theta_0,c_0)$ (where $c$ is the
recombination rate per site).  The concept of the
neighborhood of rearrangement, used in the non-recombinant algorithm,
no longer applies, since any lineage can
potentially recombine with any other, including lineages not currently
represented in the genealogy.

One approach, which we are currently investigating, is to allow two
different types of modifications:  one type which modifies branching order
and length (analogous to the modifications in the non-recombinant
version), and one which adds and removes recombinations.  Some
simplification of the full recombinant coalescence process will probably
be required.  The non-recombinant coalescent need only consider the
probability that three lineages will coalesce into two or one during an
interval of time.  In a recombinant genealogy, three lineages could
split into an unlimited number of lineages via recombination.  An
explicit solution for this infinite family of probabilities (three
lineages become four, five, ....) is probably too difficult.  A
truncated version that allows only a limited number of new recombinations to
occur during a time interval may be an adequate substitute.

{\bf Migration.}
Takahata (1985) explored the coalescent theory of populations that
exchange migrants, and found it intractable except for very small cases.
The difficulty arises in allowing for all possible combinations of
migration events.  As an alternative approach, a Metropolis-Hastings
sampler which explicitly included migration events as well as
coalescences in its genealogies could be used.  Such a sampler would
allow two types of modification to a genealogy:  a change of branching
order or length (as in the original algorithm), and a modification of
the putative migration events.  Such a search could be made fairly
efficient by a modification scheme which did not propose migration
events inconsistent with the present-day subpopulation structure,
although it would still need to search through a very large state space
of genealogies with migration.

{\bf Selection.}
General models of selection are probably too complex for the
Metropolis-Hastings approach, but some special cases can be analyzed in
this fashion.  Consider a mutation at a single site which is strongly
favored and thus rapidly goes to fixation in the population.  Sites which
are closely linked to the favored site will also tend to be dragged to
fixation with it (``hitchhiking"), while more distant sites will retain
polymorphism due to recombination.  (In non-recombining DNA a single
selected site will draw the entire molecule to fixation, providing no
information about which site was selected.)  In essence, the selected
site sees an extremely small $\Theta$ (because the only individuals who
contributed to the current population were those
carrying the site destined to be
fixed) while adjacent sites see a larger and larger $\Theta$ 
approaching the neutral $\Theta$ of the genome as a whole.  This type of
pattern should be detectable with Metropolis-Hastings sampling of
recombinant genealogies that explicitly include instances of
hitchhiking.

Another potentially tractable type of selection is balancing selection
at a specific site.  A population could be considered to consist of two
subgroups, each with a specific base at the selected site:  the two
subgroups would be maintained at constant size by natural selection.
Mutation at the key
site could move an entire sequence from one subgroup to another;
recombination could move the part of the sequence not containing the key
site.  It should be possible to detect which site is defining the
subgroups, although this will require the machinery of both
recombination and migration.

{\bf Disequilibrium mapping.}
The Metropolis-Hastings approach could be used for a type of
disequilibrium linkage mapping.  Conventional linkage mapping uses
recombinations observed in family data to locate a gene of interest with
respect to marker loci.  The resolution of conventional mapping is
limited:  the closer one is to the gene of interest, the rarer are the
informative recombinations.  Disequilibrium mapping attempts to use
information from recombinations occuring throughout the history of the
region, which should potentially make many more recombinations available
for study.

Current methods of disequilibrium mapping are based on pairwise
combinations of loci, losing some of the available information.  A
Metropolis-Hastings method with recombination could be used for mapping
by computing a likelihood function across the sampled genealogies for
each possible location of the gene to be mapped.  This will presumably
require a likelihood model for non-sequence data such as restriction
fragment polymorphisms (Felsenstein 1992c), since sequence data across the entire area to be
mapped are unlikely to be available.

One problem to be overcome is that a sample with many copies
of a rare trait (such as a genetic disease)
is likely to be quite different from a random population
sample, and this ascertainment bias must be allowed for in constructing
the Metropolis-Hastings sampler.  Furthermore, if the trait in question
is not recessive allowance must be made for uncertainty about which of
the affected individual's two chromosomes carries it.  However, even if
the estimates of recombination rate and population size are biassed by
such ascertainment issues, the assignment of the trait to a particular
chromosome region may still be made correctly.

\bigskip
{\center Discussion}
\bigskip

The Metropolis-Hastings sampler is a relatively efficient way to explore
the space of possible genealogies for a given data set. 
In preliminary simulations (Kuhner, Yamato, and Felsenstein 1995,
submitted) we used samples of 100 sequences of length 1000, performing
a total of 105,500 steps divided among 7 Markov Chains:  the entire
estimation took approximately 181 minutes on
a DECstation 5000/125, a mid-speed workstation.  The run time
does not increase quickly with increasing numbers of sequences or
sites.  Samples of over 800 sequences are feasible on 
a workstation, although many steps along the chain will be necessary
in order to adequately survey the space of plausible genealogies.

In principle the Metropolis-Hastings genealogy sampling approach can be
extended to a number of key problems, including variable population
size, recombination, migration and selection.  The more complex models
will probably demand large numbers of relatively long sequences, and
possibly multiple loci, in order
to simultaneously estimate several parameters, but the recent explosion
in sequencing technology should soon make such data sets available.
It is interesting to contemplate a ``grand unified model" of sequence
evolution incorporating all the listed processes, and perhaps insertion,
deletion, rearrangment, and interactions among sites as well.  Only time
will tell whether such a unified model is feasible.

\bigskip
{\center Software availability}
We are making available a new package, LAMARC, containing our
Metropolis-Hastings programs.  The first program, COALESCE,
implementing the Metropolis-Hastings genealogy sampler for
the simplest case (neutral non-recombining sequences in a constant-size,
randomly mating population) is available by anonymous ftp from {\it
evolution.genetics.washington.edu} in directory /pub/lamarc.  It is
written in C.  Future programs will also be available in the same
directory.

\bigskip
{\center Acknowledgements}

\bigskip

We thank the Institute for Mathematics and its Applications for inviting
us to the Workshop on Mathematical Population Genetics, supported by
funds provided by the National Science Foundation.  The research
presented here was supported by National
Science Foundation grants BSR-8918333 and DEB-9207558 and National Institute of
Health grant 2-R55GM41716-04 (all to J. F.).

\bigskip
{\center Literature Cited}

\bigskip

{\parindent=-0.2in

\medskip
{\sc Felsenstein}, J., 1992a  Estimating effective population size from
samples of sequences:  a bootstrap Monte Carlo integration method.
Genet. Res. {\bf 60:} 209-220.

\medskip
{\sc Felsenstein}, J., 1992b  Estimating effective population size from
samples of sequences:  inefficiency of pairwise and segregating sites as
compared to phylogenetic estimates.  Genet. Res. {\bf 56:} 139-147.

\medskip
{\sc Felsenstein,} J.,  1992c  Phylogenies from restriction sites, a maximum
likelihood approach.  Evolution  {\bf 46:} 159-173.

\medskip
{\sc Fu,} Y.-X., 1994  A phylogenetic estimator of effective population
size or mutation rate.  Genetics {\bf 136:} 685-692.

\medskip
{\sc Geyer,} C. J., 1991  Estimating normalizing constants and reweighting
mixtures in Markov Chain Monte Carlo.  Technical Report No. 568, School
of Statistics, University of Minnesota.

\medskip
{\sc Griffiths,} R. C., and S. {\sc Tavar\'{e}}, 1993a  Sampling theory for neutral
alleles in a varying environment.  Proc. R. Soc. Lond. B in press.

\medskip
{\sc Griffiths,} R. C., and S. {\sc Tavar\'{e}}, 1993b  Inference for the 
infinitely-many-sites model.  Genetics in press.

\medskip
{\sc Kishino, H., T. Miyata} and {\sc M. Hasegawa}, 1990  Maximum
likelihood ingerence of protein phylogeny and the origin of
chloroplasts.  J. Mol. Evol. {\bf 31:} 151-160.

\medskip
{\sc Hastings,} W. K., 1970  Monte Carlo sampling methods using Markov chains
and their applications.  Biometrika {\bf 57:} 97-109.

\medskip
{\sc Kingman,} J. F. C.,  1982a  The coalescent.  Stochastic Processes and 
Their Applications {\bf 13:} 235-248.

\medskip
{\sc Kingman,} J. F. C.,  1982b  On the genealogy of large populations.  J.
Applied Prob. {\bf 19A:} 27-43.

\medskip
{\sc Kishino,} H., and M. {\sc Hasegawa,} 1989  Evaluation of the maximum likelihood
estimate of the evolutionary tree topologies from DNA sequence data, and
the branching order in Hominoidea.  J. Mol. Evol. {\bf 29:} 170-1790.

\medskip
{\sc Kuhner,} M. K., and J. {\sc Felsenstein,} 1994  A simulation
comparison of phylogeny algorithms under equal and unequal evolutionary
rates.  Mol. Biol. Evol. in press.

\medskip
{\sc Kuhner,} M. K., J. {\sc Yamato,} and J. {\sc Felsenstein,} 1995  
Estimating effective
population size and mutation rate from
sequence data using Metropolis-Hastings sampling.
Genetics submitted.

\medskip
{\sc Metropolis,} N., A. W. {\sc Rosenbluth,} M. N. 
{\sc Rosenbluth,} A. H. {\sc Teller,} and E. {\sc Teller},
1953  Equations of state calculations by fast computing machines.  J.
Chem. Phys. {\bf 21:} 1087-1092.

\medskip
{\sc Takahata,} N., 1985  The coalescent in two partially isolated
diffusion populations.  Genet. Res. {\bf 52:} 213-222.

\medskip
{\sc Vigilant, L., M. Stoneking, H. Harpending, K. Hawkes,} and {\sc A. C.
Wilson,} 1991  African populations and the evolution of human
mitochondrial DNA.  Science {\bf 253:} 1503-1507.
 
\medskip
{\sc Viterbi,} A. J., 1967  Error bounds for convolutional codes and an
asymptotically optimum decoding algorithm.  IEEE Trans. Inform. Theory
{\bf IT-13:} 260-269.

\medskip
{\sc Yang,} Z., 1993  Maximum-likelihood estimation of phylogeny from
DNA sequences when substitution rates differ over sites.  Mol. Biol.
Evol. {\bf 10:} 1396-1401.
}

\bigskip
{\center Figures}
\bigskip

Figure 1:  A recombinant genealogy.  Recombination has occured between
sites 9 and 10.  The upper genealogy represents the entire sequence; the
two lower genealogies show its decomposition into multiple
non-recombinant genealogies, each applying to a subset of the sites.

\end{document}

