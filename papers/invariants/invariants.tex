\documentstyle{article}
\def\baselinestretch{1.5}
\begin{document}
\centerline{Counting phylogenetic invariants in some simple cases} 

\bigskip
\centerline{Joseph Felsenstein}
\bigskip 

\medskip
\centerline{Department of Genetics SK-50}
\medskip 

\centerline{University of Washington}
\medskip
\centerline{Seattle, Washington 98195}
\bigskip 

\bigskip 

\centerline{{\it Running Headline:}  Counting Phylogenetic Invariants}
\bigskip 

\bigskip 

\centerline{ABSTRACT}
\medskip

An informal degrees-of-freedom argument is used to count the number of 
phylogenetic invariants in cases where we have 3 or 4 species and can assume a 
Jukes-Cantor model of base substitution with or without a molecular clock.
A number of simple cases are treated 
and in each the number of invariants can be found.  Two new classes of 
invariants are found: non-phylogenetic cubic invariants testing independence 
of evolutionary events in different lineages, and linear phylogenetic 
invariants which occur when there is a molecular clock.  Most of the linear 
invariants found by Cavender (1989) turn out in the Jukes-Cantor case to be 
simple tests of symmetry of the substitution model, and not phylogenetic 
invariants.
\bigskip 

\centerline{1. Introduction}
\medskip 

The method of phylogenetic invariants, also known as ``evolutionary parsimony", 
has been introduced into the field of phylogenetic inference by Cavender 
(Cavender and Felsenstein, 1987) and by Lake (1987).  The invariants are 
polynomial expressions (quadratic in Cavender's case, linear in Lake's) in the 
expected frequencies of different patterns of characters states.  They are 
invariants if they have the same value (usually zero) for all phylogenies of a 
given tree topology.  I define {\it phylogenetic invariants} as the ones that 
have the same value for all phylogenies of one tree topology, but have a
different value for at least one tree of a different topology.  They are
usually 
constant only in one tree topology.  Non-phylogenetic invariants have the
same value in all phylogenies.  Testing whether phylogenetic invariants have
the expected value is useful as a test of the tree topology. 

In this paper I will count the number of invariants that 
exist in certain cases, those with a symmetric model of change of character 
state.  The cases I will consider have four character states (the four 
nucleotides A, C, G, and T) and four species.  The model of nucleotide 
substitution considered is that of Jukes \& Cantor (1969).  I will also 
consider similar models with two and three states.
\bigskip 

\centerline{2. Definitions and Notation}
\medskip 
     
Suppose that we have four different species for which we have nucleotide 
sequences, and consider $s$ sites at which they can be correctly aligned, and 
in which the nucleotides are available in all four sequences.  We will assume 
that the process of evolution occurs independently at each site, though of 
course not independently in the four species.  The likelihood of the sequences 
is then a product over sites. If $p_{ijkl}$ is the probability, at a site, of 
observing the nucleotides $i$, $j$, $k$ and $l$ in the four sequences, and if 
$b_{ij}$ is the base observed at site $j$ of species $i$, the likelihood is: 

\begin{equation} % 1
         L  =   \prod\limits_{i=1}^s p_{b_{1i}b_{2i}b_{3i}b_{4i}}. 
\end{equation}

We can rearrange the terms in this product so that terms that 
have the same four-tuple of bases are adjacent.  This leads to the alternate 
form:
\begin{equation} % 2
         L   =   \prod\limits_{ijkl} p_{ijkl}^{n_{ijkl}}
\end{equation}
where the indices $i$, $j$, $k$, and $l$ each run over the four bases in the
set $\{A, C, G, T\}$ and $n_{ijkl}$ is the number of terms in (1) that have the 
bases $i$, $j$, $k$, and $l$.  The sum of the $n_{ijkl}$ must be the total 
number of sites $s$. 

The 4-tuple of bases $ijkl$ will be called, in agreement with Cavender (1978) a 
{\it pattern}. There are 256 possible patterns, AAAA, AAAC, ... TTTT, if we 
ignore (as we do) ambiguous nucleotides.  $n_{ijkl}$ is thus the observed 
number of occurrences of pattern $ijkl$. 

Note that the probabilities $p_{ijkl}$ are functions of the tree topology and 
the lengths of branches, plus whatever other parameters exist in the model of 
base change.  I have presented here only the case of four species, but the 
three-species case is entirely analogous, leading to:

\begin{equation} % 3
         L   =   \prod\limits_{ijk} p_{ijk}^{n_{ijk}}
\end{equation} 

It should be immediately clear that the set of $n_{ijkl}$ are sufficient 
statistics for estimation of the phylogeny of the species and the testing of 
assertions about the model of nucleotide substitution.  In the three-species 
case the sufficient statistics are of course the $n_{ijk}$.
\bigskip 


\centerline{3. The model of base substitution}.
\medskip

The results of this paper depend on a symmetric model of base
substitution, that of Jukes \& Cantor (1969).  A natural question, left
unanswered here, is to what extent these results would generalize to
models with fewer symmetries, notably the 2-parameter model of Kimura (1980).
The Jukes-Cantor model is the simplest possible symmetric model.
The probabilities of base change in a given branch of a phylogeny are given by
the table:
\begin{equation} % 4
\begin{array}{l l c c c c}
              &to:&  A &   C &   G &   T   \\
         from:   & \\
              &A  & 1-q & q/3 & q/3 & q/3 \\
              &C  & q/3 & 1-q & q/3 & q/3 \\
              &G  & q/3 & q/3 & 1-q & q/3 \\
              &T  & q/3 & q/3 & q/3 & 1-q
\end{array}
\end{equation}

where $q$ is a parameter which depends on the product of the length of the
branch in time ($t$) and the substitution rate ($u$) per nucleotide in that branch:
\begin{equation} % 5
      q   =  {3 \over 4} (1  -  e^{- {4 \over 3} u t})
\end{equation}
Basically $q$ is the net probability of change in that branch, and is the same
no matter what the current nucleotide.  If a nucleotide changes, it has an
equal probability of changing to each of the other three nucleotides.
Under the Jukes-Cantor model the equilibrium distribution of nucleotides
(the expected nucleotide composition) is 0.25 : 0.25 : 0.25 : 0.25.  The
largest biologically reasonable value of $q$ is 3/4.  We assume in all cases
that the evolutionary process has reached equilibrium before the start of the
divergence of the species.
\bigskip

\centerline{4. A simple case: 3 species and no clock}
\medskip

The meaning of invariants will be clearest in the simplest cases.  It is
useful to start with three species, a Jukes-Cantor model, and no
molecular clock.  The ``molecular clock" is in this context simply the
assertion that the probability of base substitution ($u$) is constant per unit
time and the same in all lineages.  With three species and four nucleotides
there are 64 patterns, AAA, AAC, ... TTT.  It should immediately be apparent
that the symmetry of the Jukes-Cantor model will leave the probability of
any given set of data, as expressed by the equation
(3), unchanged if we replace all A's in the data by C's and all C's by A's,
or for that matter if we carry out any other permutation of the four bases.
Thus if $\sigma$ is the permutation, so that $\sigma_b$ is the base into
which base $b$ is changed by the permutation,

\begin{equation} % 6
     p_{\sigma_i \sigma_j \sigma_k} = p_{ijk}
\end{equation}

for all bases $i$, $j$, $k$, and $l$, whether or not those are distinct.  An analogous
result of course holds for the four-species case.

\medskip

{\it Symmetry tests.}  The probability of the pattern AAC
on a given tree will be
the same as the probability of pattern GGA, and similarly for any pattern
of the form xxy, where x and y stand for any two distinct nucleotides.  This
argument allows us to see that there are in fact only five types of patterns:
xxx, xxy, xyx, yxx, and xyz, where x, y, and z are distinct nucleotides.  We
will call these classes of patterns {\it pattern types}.  Type xxx consists of
patterns AAA, GGG, CCC, and TTT.  All four of these have equal expected
frequencies.  One test of the symmetry which the Jukes-Cantor model predicts
is to test whether these four
patterns are significantly unequal in frequency.  This can easily be
done by a chi-square test of the equality of the numbers of times the four
patterns occur.  The test has 3 degrees of freedom.

There are similar tests within each pattern type.  Table 1 shows an
accounting of the different pattern types, how many patterns each
contains, and thus how many degrees of freedom are available within each
pattern type for testing the symmetry of the model of base substitution.
There are 64 patterns in all.  The pattern frequencies thus have 63
degrees of freedom, which is equivalent to the statement that the
64-dimensional space of expected pattern frequencies is constrained to a
63-dimensional subspace, namely those that add up to 1 (they are also
confined to a simplex in that subspace, namely the sets of frequencies
that have no negative frequencies).

The accounting
of symmetry test degrees of freedom shows that there are 59 equations,
in fact linear equations, which are consequences of the symmetry. For
example,  $p_{AAA} = p_{CCC}$ and $p_{ACG} = p_{AGT}$ are two of them.  If the
expected pattern frequencies satisfy these 59 linear equations as
well, they must lie in a 64 - 1 - 59 = 4-dimensional linear subspace.
The 4-dimensional subspace in fact corresponds precisely to the 5
pattern types, less one for the fact that the total frequencies of pattern
types adds to 1.

The sufficient statistics for estimating the phylogeny in this case are not the
observed numbers of the 64 patterns, but the observed numbers of
occurrences of the 5 pattern types.  Let us represent by $n_{xxx}$ the
frequency of any one of the patterns of type xxx, and by $N_{xxx}$ the
total frequency of all patterns of type xxx, and analogously for the
other patterns.  Let $p_{xxx}$ and $P_{xxx}$ be the expected frequencies of
one pattern of type xxx and the total expected frequency of all patterns
of type xxx, and analogously for the other patterns.

We can note that in this model the expected frequencies of the five
pattern types can be written as functions of the unknown net probabilities of
change in the three branches of the unrooted tree, which we will call $q_1$,
$q_2$, and $q_3$:

\medskip
\begin{equation} % 7
P_{xxx}   =   (1-q_1)(1-q_2)(1-q_3) + q_1q_2q_3/9
\end{equation}
\medskip
\begin{equation} % 8
P_{xxy}   =   (1-q_1)(1-q_2)q_3  +  1/3\; q_1q_2(1-q_3)  +  2/9\; q_1q_2q_3
\end{equation}
\medskip
\begin{equation} % 9
P_{xyx}   =   (1-q_1)q_2(1-q_3)  +  1/3\; q_1(1-q_2)q_3  +  2/9\; q_1q_2q_3
\end{equation}
\medskip
\begin{equation} % 10
P_{yxx}   =   q_1(1-q_2)(1-q_3)  +  1/3\; (1-q_1)q_2q_3  +  2/9\; q_1q_2q_3
\end{equation}
\medskip
\begin{equation} % 11
P_{xyz}   =   2/3\; q_1q_2(1-q_3)  +  2/3\; q_1(1-q_2)q_3\\
              + \; 2/3\; (1-q_1)q_2q_3  + 2/9\; q_1q_2q_3
\end{equation}
\medskip

The expected frequencies of the five pattern types add up to 1.  Thus we
have four equations in three unknowns.  This means that there is one
algebraic relationship between the five quantities.  

After some tedious algebra (for which see Appendix 1), one can find this
relationship, a cubic polynomial equation:

\begin{equation} % 12
\begin{array}{l c c}
\medskip
3 \; (2 P_{xxx} - 2 P_{xxy} - 2 P_{xyx} - 2 P_{yxx} + 1)^2& &\\

 - [4(P_{xxx} + P_{xxy}) -  1] [4(P_{xxx} + P_{xyx}) - 1] [4 (P_{xxx} + P_{yxx}) - 1] &  = &  0
\end{array}
\end{equation}

I suspect that this cubic equation is a consequence of the independence of
substitution in different lineages.

With three species, the Jukes-Cantor model, the expressions (7)-(11) do
not depend on where the root of the tree is, and there is only one
possible unrooted tree topology.  The cubic polynomial is an invariant
for this case, in that it is an expression in the expected pattern type
frequencies which has the same value for all trees of a given topology,
whatever their branch lengths.  In fact, since the placement of the root of
the tree does not affect the pattern frequencies, and since there is
only one possible unrooted tree topology, this cubic invariant is not a
phylogenetic invariant.  We have thus accounted
for all the degrees of freedom in this case, and find invariants, but of course
no phylogenetic invariants.
\bigskip

\centerline{5. The Jukes-Cantor model with 3 species and a clock.}
\medskip

The first signs of phylogenetic invariants occur when we impose the
constraint of a molecular clock.  If the tree topology is
the second tree in Fig. 1, this corresponds to requiring that $q_1 = q_2$.
and that $q_3 > q_1$.  There are still 64 patterns and 63 degrees of
freedom, and still 59 of these which are accounted for by the symmetries
of the Jukes-Cantor model.  Of the 4 remaining degrees of freedom, 2 of them
are accounted for by the branch lengths of the tree.  With a molecular clock,
the constraint that $q_1 = q_2$ implies that there are only two independent
branch lengths ($q_1$ and $q_3$).

One degree of freedom is still accounted for by the cubic equation (12),
which still holds in this case, as the case of a clock is a subcase of
the preceding one.  This leaves us with one degree of freedom unaccounted for.
We have 5 expected pattern frequencies which must sum to 1, and two parameters,
implying that two equations can be written in the expected pattern type
frequencies.  One of these is (12).  The other is not hard to find.  The
equality of the branch lengths $q_1$ and $q_2$ implies that patterns xyx and
yxx are expected to be equally frequent.  Examination of equations (8) and
(9) verifies that if $q_1 = q_2$ then $P_{xyx} = P_{yxx}$.
This is the missing invariant.  It is not simply a consequence of
the symmetry of
the model of base substitution or of the independence of substitutions in
different branches of the tree, for it will hold in models that lack either
of these, as long as a clock can be assumed.  This invariant is a phylogenetic invariant,
as in the third tree in Fig. 1 it is no longer true that $P_{xyx} = P_{yxx}$,
but now instead
$P_{xyx} = P_{xxy}$.  We have thus identified a phylogenetic invariant, and a
linear one at that.  Strictly speaking, the invariants are the expressions
\begin{equation} % 13
         C_1   =   P_{yxx} - P_{xyx},
\end{equation}

\begin{equation} % 14
         C_2   =   P_{xyx} - P_{xxy},
\end{equation}
and
\begin{equation} % 15
         C_3   =   P_{xxy} - P_{yxx}.
\end{equation}

Note that $C_1 + C_2 + C_3 = 0$.  For the second tree shown in Fig. 1,
$C_1 = 0$, and we must then have $C_2 + C_3 = 0$.  For each of the other two
bifurcating tree topologies, there are similar relationships with $C_2 = 0$ or
$C_3 = 0$.
\bigskip

\centerline{6.  Jukes-Cantor model with 4 species and no clock.}
\medskip

When we reach 4 species things become more complicated.
There are 256 patterns of nucleotides.  The situation is shown in Table
2.  Taking into account the
exchangeability of the four nucleotides, there are now 15 pattern types.  This
means that after the symmetry invariants have been taken out, there must
remain 15 degrees of freedom, so that there are fully 241 degrees of freedom
accounted for by symmetry of the nucleotides.  These 15 degrees of freedom can
be reduced by one since the expected pattern frequencies must add to 1.

\medskip

There is, in an unrooted 4-species tree, one parameter per branch and there
are five branches.  This leaves 15-1-5 = 9 degrees of freedom.  Fortunately,
we can make use at this point of invariants found by Cavender (Cavender \&  
Felsenstein, 1987), Lake (1987), and Drolet \& Sankoff (1990) to account
for some of these.
\bigskip

a. Cavender's Invariants
\medskip

Cavender investigated the a model of two states, 0 and 1,
and found for each of the three possible unrooted tree topologies with four
species and no clock that there were two quadratic expressions in the expected
frequencies that were phylogenetic invariants.  In the present case we can
group the four bases into two groups of two in any way (it does not matter
how because of the symmetry of the bases).  We may, for example, code bases
into R and Y (purine and pyrimidine).  R and Y will evolve as two symmetric
states, fitting Cavender's model exactly.  We can then classify the 256 site
patterns into sixteen classes: RRRR, RRRY, ... YYYY.

The symmetry between
the Y and R symbols reduces these further to eight: 0000, 0001, 0010, 0011,
0100, 0101, 0110, and 0111, where 0 and 1 are place holders of which one 
stands for an R and the other a Y.  The
expected frequencies of these eight classes of patterns must satisfy Cavender's
two phylogenetic invariants (his K and L invariants), as the evolution of
states R and Y follows his assumptions exactly.  We shall here call the
frequencies of these eight pattern types $S_{0000}, S_{0001}, S_{0010}, 
S_{0011}, S_{0100}, S_{0101}, S_{0110}$, and $S_{0111}$, as they
aggregate the patterns into types in a way different from the classes
whose frequencies are indicated above by the $P$'s.

For the first tree topology in Fig. 2, Cavender's K-invariant is
\begin{equation} % 16
K_1 = (S_{0100}-S_{0111})(S_{0010}-S_{0001}) - (S_{0110}-S_{0101})(S_{0000}-S_{0011})
\end{equation}
and his L-invariant is:

\begin{equation} % 17
L_1  =  (S_{0001} + S_{0010})(S_{0100} + S_{0111}) - (S_{0000} + S_{0011})(S_{0101} + S_{0110})
\end{equation}

Both of these invariants are zero, and both are phylogenetic invariants.  It
is worth noting that the Cavender K invariant can be considered to be a
consequence of the four-point metric condition of Buneman (1974).   Buneman
pointed out that for a tree of this topology if there is a distance
$d_{ij}$ that is additive along branches of the first tree in Fig. 2,
it must satisfy
\begin{equation} % 18
   d_{14} + d_{23} = d_{13} + d_{24}.
\end{equation}

In the derivation of Cavender's result if we note that the branch lengths
are additive, and that a branch length $t$ may be expressed in terms of
the probability $D$ that the states of the species at the two ends of the
branches are different as
\begin{equation} % 19
   t  =  - {1 \over 2} \; ln \; (1 - 2D).
\end{equation}

If $D_{ij}$ is the probability that species $i$ differs in state from
species $j$, in the two-state case

\begin{equation} % 20
   D_{14} = S_{0001} + S_{0011} + S_{0101} + S_{0111}
\end{equation}

\begin{equation}
   D_{23} = S_{0010} + S_{0011} + S_{0100} + S_{0101}
\end{equation}

\begin{equation}
   D_{13} = S_{0010} + S_{0011} + S_{0110} + S_{0111}
\end{equation}

\begin{equation}
   D_{24} = S_{0001} + S_{0011} + S_{0100} + S_{0110}
\end{equation}

We can express the $D_{ij}$ in terms of the $S$'s in this way, and use
these and equation (19) to express the total branch lengths between species $i$
and $j$ in terms of the $S$'s.  Since these total branch lengths can be used
in place of the $d_{ij}$ to satisfy Buneman's condition, we end up with an
expression in the $S$'s.  This turns out to be precisely Cavender's $K$ 
invariant.

One might imagine that another classification of the four bases into two
sets of two bases each would lead to a different pair of invariants based on
Cavender's invariants.  This is not true, since the symmetry invariants
guarantee that the invariants are precisely equal no matter which two bases
are called Y and which two R.

\bigskip

b. Lake's Invariants
\medskip
         
Lake (1987) found two linear invariants in a model of base change which had
balanced transversions (so that if an A changed, it was equally likely to
change to a C or a T).  The Jukes-Cantor model has this property, among
others.  Thus Lake's two linear invariants must also
apply to the present model.

In the present notation, Lake's invariants are

\begin{equation} % 24
{2 \over 3}P_{xyxy} + {1 \over 3}P_{xyzw} - {1 \over 3}P_{xyxz} - {1 \over 3}P_{xyzx} = 0
\end{equation}

and

\begin{equation} % 25
{2 \over 3}P_{xyyx} + {1 \over 3}P_{xyzw} - {1 \over 3}P_{xyzx} - {1 \over 3}P_{yxxz} = 0
\end{equation}

Taking Cavender's and Lake's invariants
into account reduces the 9 degrees of freedom by 4 so that we have 5 degrees
of freedom to account for.

\bigskip

c. Three-species cubic invariants
\medskip

In the three-species Jukes-Cantor case we found one cubic invariant.  In the
present case we can always consider three of the four species and ignore the
remaining one.  The 256 patterns then reduce to 64 in the obvious way:
if the first species is being ignored, the pattern AAA refers to any pattern
which has A in all of the last three species.  Its
expected frequency is the sum of
the frequencies of AAAA, CAAA, GAAA and TAAA.  The expected frequencies of
these 64 classes will satisfy the cubic polynomial (12).  There are four
different
ways in which we can drop one of the species (one for each species we could
drop), hence four such cubic invariants.  None of
them is a phylogenetic invariant.  It will not take the reader long to satisfy
themself that these four quantities are independent.  Each depends on a
different three-species marginal distribution, and none of those can
distributions can be computed from each other.  We have thus reduced
the number of degrees of freedom from 5 to 1.

\bigskip

d. Drolet-Sankoff quadratic invariant
\medskip

That one remaining degree of freedom is the four-state quadratic invariant
discovered by Drolet \& Sankoff (1989).  They investigated the case of
four species, without a clock, and with the a symmetric model of change among
$n$ states.  The Jukes-Cantor model is the $n=4$ case of the one they consider.
The quantity they found is a phylogenetic invariant which is quadratic.
This too must be satisfied by the expected frequencies in the present case.
Drolet and Sankoff's first quadratic phylogenetic invariant is:

\begin{equation} % 26
F_2 - F_3,
\end{equation}

where

% 27
\begin{equation}
\begin{array}{c c l}
F_2 & = & [4(P_{xxxx} + P_{xyxy} + P_{xyxz} + P_{xyzy}) - 1] \times \\
    &   &  [4(P_{xxxx} + P_{xyxy} + P_{xyyy} + P_{xxyx} + P_{xyxx} + P_{xxxy}) - 1]\\
    &   & +\; 4(P_{xyyy} + P_{xxyx} - P_{xyxz}) 4(P_{xyxx} + P_{xxxy} - P_{xyzy})
\end{array}
\end{equation}
and

% 28
\begin{equation}
\begin{array}{c c l}
F_{3} & = & [4(P_{xxxx} + P_{xyyx} + P_{xyzx} + P_{xyyz}) - 1] \times \\
      &   &  [4(P_{xxxx} + P_{xyyx} + P_{xyyy} + P_{xxxy} + P_{xxyx} + P_{xyxx}) - 1]\\
       &   &    +\; 4(P_{xyyy} + P_{xxxy} - P_{xyzx}) 4(P_{xxyx} + P_{xyxx} - P_{xyyz})
\end{array}
\end{equation}

for which the invariant is zero.  They also found another quadratic invariant.  
One might think that this is one too many.  Actually, it is not different from
the L invariant of Cavender.  In the case of the Jukes-Cantor model it can be
shown (Appendix 2) that when the Drolet-Sankoff L invariant has its expected
value, and when the symmetry invariants do also, that the Cavender L invariant
must also have its expected value.

It is interesting and important to note that
we have now completely accounted for the degrees of freedom:
\medskip

\begin{tabular}{r l}
              5  &   branch length parameters\\
              1  &   Drolet-Sankoff quadratic phylogenetic invariant\\
              2  &   Lake linear phylogenetic invariants \\
              2  &   Cavender 2-state quadratic phylogenetic invariants\\
              4  &   three-species cubic invariants\\
            241  &   linear invariants testing symmetry of base substitution\\
              1  &   since the expected frequencies add to 1\\
           ----- & \\
            256 &
\end{tabular}
\medskip

Note that only 5 of these 256 degrees of freedom are phylogenetic invariants.
\bigskip

\centerline{7.  Jukes-Cantor model with four species and a molecular clock.}
\medskip

When we constrain the preceding case so that the tree is clocklike, the
picture changes slightly.  The 5 branch lengths are replaced by 3
divergence times.  All the other invariants continue to be zero, as this
case is a subcase of the preceding one.  Thus we have 2 degrees of
freedom unaccounted for.  These must test the clockness of the tree,

Fig. 3 shows the two forms of possible clocklike unlabelled
tree topologies.  The 15 possible bifurcating tree topologies are all of
one or the other of these two kinds.  The enumeration of degrees of freedom is
the same as before except that the five degrees of freedom for branch lengths
are replaces by 3 for branch lengths and 2 for the phylogenetic invariants 
for clockness.
\medskip

\begin{tabular}{r l}
              3  &   branch length parameters\\
              2  &   linear phylogenetic invariants for clockness\\
              1  &   Drolet-Sankoff quadratic phylogenetic invariant\\
              2  &   Lake linear phylogenetic invariants \\
              2  &   Cavender 2-state quadratic phylogenetic invariants\\
              4  &   three-species cubic invariants\\
            241  &   linear invariants testing symmetry of base substitution\\
              1  &   since the expected frequencies add to 1\\
           ----- & \\
            256  &
\end{tabular}
\medskip

(and a partridge in a pear tree).

There is one surprise in the clock case.  For the first tree topology in
Fig. 3, consideration of the symmetries will immediately suggest that the
following are invariants:

\medskip

\begin{equation} % 29
P_{xyxx} = P_{yxxx}, \\
\end{equation}
\begin{equation}
P_{xyxy} = P_{xyyx}, \\
\end{equation}
\begin{equation}
P_{xyxz} = P_{xyyz}, \\
\end{equation}
\begin{equation}
P_{xyzx} = P_{xyzy}.
\end{equation}


The problem is that there are too many of them.  We are supposed to have
2 degrees of freedom to test clockness, not 4.  The dilemma could be resolved
if some of these invariants were not independent, if they were implied by
combinations of others.  In fact, this is the case.  We can use (29)-(32)
to show straightforwardly that when these hold, Lake's two linear invariants
(24) and (25) are equal.  We can also show that Cavender's K invariant must
equal 0, and we can also show that the three species cubic invariant for
species 1, 3, and 4 necessarily equals that for species 2, 3, and 4.  These
are explained in Appendix 3.  Therefore
equations (29)-(32) represent only one new clock invariant, not four.  This is
one too few clock invariants.  In Appendix 3 it is demonstrated that there is
one more clock invariant:

\begin{equation} % 33
2 P_{xxyx} + P_{xyxy} + P_{xyyx} + P_{yxzx} + P_{xyzx} - P_{xyxx} - P_{yxxx}
- 2 P_{xxyy} - 2 P_{yzxx} = 0
\end{equation}

which is written more simply in another form in that Appendix.

For the second tree topology
in Fig. 3, the same approach immediately suggests 6 invariants: the ones
in (29)-(32) plus two more:

\medskip

\begin{equation}
P_{xxxy} = P_{xxyx}
\end{equation}
\begin{equation}
P_{xyxz} = P_{xyzx}
\end{equation}

These can also be used to prove the equivalence of the two Lake linear
invariants, and to prove that the Cavender K invariant is zero.  They also
prove the equivalence of two pairs of three-species cubic invariants, the
one mentioned above plus the invariants for species 1, 2 and 3 and for 1, 2,
and 4.  This leaves us with two clock invariants.  Equation (33) is not an
invariant for this tree topology.

\bigskip

\centerline{8. Cavender's multiple linear invariants}

\medskip

Cavender (1989) has found all linear invariants in a four-species case far more
general than the present model.  The Jukes-Cantor 4-species case presented 
here is a special case of the model he considers.  The present calculations
shed some light on his invariants.  Without an evolutionary clock we have
243 linear invariants.  241 of them are symmetry tests, and the two of those 
that are phylogenetic invariants are the Lake invariants.  Cavender
finds 68 linear invariants.  In the
present case most of these correspond to the symmetry tests.  In the
Jukes-Cantor
case they provide no information about the phylogeny.  The pattern 
frequencies are continuous functions of the parameters of Cavender's model.
We can invoke continuity to argue that when Cavender's model is near the
Jukes-Cantor model, that most of his linear invariants will have very little
information on the phylogeny.  Most of the phylogenetic information expressed
in the linear invariants will thus be in the Lake invariants, except possibly
when the model is far from the Jukes-Cantor assumptions.

\bigskip

\centerline{9. Properties of invariants in different cases}
\medskip

It is useful to tabulate a number of properties of the invariants.  We
have seen that some invariants (such as Cavender's K and L) are present
in models that have two states, while others (such as Lake's linear
invariants) are present only when there are 4 or more states.  In the table
below we call this the {\it state level}.  Invariants also differ according
to how many species must be present in the tree before they exist.  The
cubic invariants discussed above are present whenever there are 3 or more
species, but the others all require 4 species.  We call this number the
{\it species level}.  The invariants are all polynomials in the expected
pattern frequencies; they differ according to the degree of the polynomial,
which we indicate by {\it degree}.  Some are phylogenetic, some not.  Finally,
they differ in one more way.  If we consider the patterns as being in a 4-way
table, we can compute the various marginal sums of this table.  Some 
invariants can be computed using only these marginal sums.  For example
Cavender's K can be computed using only two-species marginals, but Lake's
linear invariants cannot be computed from marginals.  We call this the
{\it interaction level} of the invariant.

Here is a table of these properties, for the 4-species case with a clock:
\medskip

\centerline{
\begin{tabular}{l c c c c c}
          & State  & Species & & &  \\
\medskip
Invariant & level &  level & Degree & Phylogenetic & Interaction level\\
& & & & &\\
Symmetry       &     2       &       2       &   1    &      no      &   4 \\
Clock          &     2       &       3       &   1    &      yes     &   2 \\
Cubic          &     3       &       3       &   3    &      no      &   3 \\
Lake           &     3       &       3       &   1    &      yes     &   4 \\
Cavender K     &     2       &       4       &   2    &      yes     &   2 \\
Cavender L     &     2       &       4       &   2    &      yes     &   4 \\
Drolet-Sankoff &     3       &       4       &   2    &      yes     &   4 \\
\end{tabular}
}
\medskip
 
The table gives an incomplete picture.  In some cases we have denoted an 
invariant as present for a given number of species or a given number of states
even though the number of that class of invariants rises as the number of
states or species rises.  Thus Cavender's K invariant, a single invariant, is
present whenever there are two or more species.  By contrast, one invariant
related to Lake's linear invariants is present when there are 3 states and
4 species, but when the number of species increases to 4 there are then two
Lake invariants.  The three-state Lake-like invariant is

\begin{equation}
P_{xyxy} + P_{xyyx} - P_{yxzx} - P_{yxxz} = 0
\end{equation}

as can be verified by exact calculation of the probabilities of these
four pattern types as functions of the branch lengths.

Table 3 gives an accounting of invariants with different numbers of species
and different numbers of states with a clock.  The corresponding table without 
the clock is the same, except that the degrees of freedom for the clock must
be added to those for branch lengths, so that entries with 3 degrees of
freedom for branch lengths and 2 for clock invariants have instead 5 degrees
of freedom for branch lengths.

\bigskip
\centerline{10. Accounting for all degrees of freedom}
\medskip

The present study accounts for all of the degrees of freedom in the
Jukes-Cantor cases with 2, 3, or 4 species and 2, 3, or 4 states.  In the
absence of a molecular clock, no new phylogenetic invariants have been found.
In the presence of the clock the linear clock phylogenetic invariants have
been found.  However we must be cautious about the notion of degrees of
freedom in the present case.  We have, for example, 256 pattern frequencies
predicted by 5 branch lengths, but it is not immediately obvious that this
means that there are 251 equations in the pattern frequencies if they are
the ones predicted by the model.  This is because the notion of degrees of
freedom applies only to linear equations, and the present equations include
quadratics and cubics.  It is possible that there are more invariants to be
found.  I suspect not, but cannot prove this.
 
It would be of interest to have an analysis similar to the present one for
the case of Kimura's (1980) two-parameter model.  While the inequality of
transition and transversion rates makes few realistic models of nucleotide
substitution close to the Jukes-Cantor model, more might be close to Kimura's
model, which allows for this inequality.  The conclusions from the Kimura
model as to which 
invariants contain the information about the phylogenies might thus be much 
closer to being correct in more realistic models.
\bigskip

\centerline{11. Likelihood, parsimony, and invariants}
\medskip

If we were to test all invariants at once for having their desired values,
this would amount to a test whether the observed pattern frequencies were in
the low-dimensional subspace defined by varying all branch length parameters
and for each computing the expected pattern frequencies.  For example, for
the 4-species case without a clock, the subspace is 5-dimensional, as
there are then 5 branch lengths.  We have not presented such a test; its
full elaboration is a matter for future work.  However a straightforward
approach would be to take the likelihood ratio between the best fitting
arbitrary expected frequencies, which will be the same as the observed
frequencies, and the best fitting expected frequencies from a phylogeny.
For any one tree topology twice the log of the likelihood ratio should be
distributed as $\chi^2$ with $255 - 5 = 250$ degrees of freedom.  This is an
asymptotic distribution, valid as the number of sites becomes large.

One difficulty with this neat picture is that we are not simply finding the
best-fitting point in the 5-dimensional subspace defined by one tree topology,
but are picking the best tree from three different tree topologies.  It is not
clear whether there is some way to correct for this.  If the three tests
were statistically independent we could do so by a Bonferroni correction for
multiple tests.

A more serious issue is how to compare different tree topologies, when we are
willing to assume that one or another of them provides a correct model for the
data.  We cannot do a simple likelihood ratio test, as the hypotheses are
not nested one within another.  It is in this case that the strengths of the
invariants approach are clearest.  Many of the invariants, we have seen, test
the symmetries of the model.  If we assume that symmetry, we can focus our
test on the phylogenetic invariants, and will not lose power by wasting effort
testing those symmetries.  The different invariants test somewhat different
aspects of the model, and this allows us to have a clearer idea what is being
accepted and what rejected.  The linear invariants are readily tested
(Lake, 1987), and the Cavender L quadratic invariants is also (Cavender \&
Felsenstein, 1987). Drolet \& Sankoff (1990) have given expressions for the
variances of the other quadratic invariants and pointed out their asymptotic
normality.

We do not yet have a complete picture of the statistical testing of invariants.
What is clear is that they provide a more precise picture of the different
kinds of evidence our data provides about tree topologies, branch lengths, and
departures from the model.  Although invariants can be related to parsimony
(Lake, 1987), they seem to me much more naturally related to likelihood methods,
providing as they do an anatomical structure of the implications of the data.

\newpage

\centerline{Acknowledgments}
\bigskip

I am indebted to James Cavender and David Sankoff for discussions of some
of these issues and to James Cavender for comments on an earlier version of
the manuscript.  This research was supported by NSF grant BSR-8614807 and by
NIH grant 5-R01 GM41716.

\newpage

\centerline{REFERENCES}
\bigskip
 
Buneman, P.  1974. The recovery of trees from measurements of dissimilarity.
pp. 387-395 in {\it Mathematics in the Archaeological and Historical Sciences},
ed. F. R. Hodson, D. G. Kendall, and P. Tautu.  Edinburgh University Press,
Edinburgh.
\medskip

Cavender, J. A.  1978.  Taxonomy with confidence. {\it Mathematical Biosciences}
{\bf 40}: 271-280.
\medskip

Cavender, J. A.  and J. Felsenstein.  1987.  Invariants of phylogenies
in a simple case with discrete states.  {\it Journal of Classification}
{\bf 4}: 57-71.
\medskip

Cavender, J. A.  1989.  Mechanized derivation of linear invariants.
{\it Molecular Biology and Evolution}  {\bf 6}: 301-316.
\medskip

Drolet, S. and D. Sankoff.  1990.  Quadratic tree invariants for
multivalued characters.  {\it Journal of Theoretical Biology} {\bf 144}:
117-129.
\medskip

Jukes, T. H. and C. R. Cantor. 1969. Evolution of protein molecules.
pp. 21-123 in {\it Mammalian Protein Metabolism III}, ed. H. N. Munro.
Academic Press, New York.
\medskip

Kimura, M.  1980.  A simple method for estimating evolutionary rate of
base substitutions through comparative studies of nucleotide sequences.
{\it Journal of Molecular Evolution} {\bf 16}: 111-120.
\medskip

Lake, J. A.  1987.  A rate-independent technique for analysis of nucleic
acid sequences: evolutionary parsimony.  Molecular Biology and Evolution
4: 167-191.
\medskip

\newpage

\centerline{Table 1}

\centerline{The pattern types with a Jukes-Cantor model with 3 species}
\centerline{and no molecular clock.}
\bigskip

\begin{tabular}{c c c c}
\medskip
               & Pattern type &   Patterns  &   Symmetry d.f.\\
                   & xxy     &      12      &       11\\
                   & xyx     &      12      &       11\\
                   & yxx     &      12      &       11\\
                   & xyz     &      24      &       23\\
                           & &     ---      &      ---\\
                    & Total   &     64       &      59
\end{tabular}

\newpage

\centerline{Table 2}

\centerline{The pattern types with a Jukes-Cantor model with 4 species}
\centerline{and no molecular clock.}
\bigskip

\begin{tabular}{c c c}
\medskip
            Pattern type  &    Patterns   &    Symmetry d.f.\\
               xxxx    &          4       &         3\\
               xxxy    &         12       &        11\\
               xxyx    &         12       &        11\\
               xyxx    &         12       &        11\\
               yxxx    &         12       &        11\\
               xxyy    &         12       &        11\\
               xyxy    &         12       &        11\\
               xyyx    &         12       &        11\\
               xxyz    &         24       &        23\\
               xyxz    &         24       &        23\\
               xyzx    &         24       &        23\\
               yxxz    &         24       &        23\\
               yxzx    &         24       &        23\\
               yzxx    &         24       &        23\\
               xyzw    &         24       &        23\\
                       &        ---       &       ---\\
                       &        256       &       241
\end{tabular}

\newpage

\centerline{Table 3}

Non-symmetry invariants with different numbers of states and of species in the case of a 
molecular clock.  The corresponding table for no clock is the same except
that the degrees of freedom for clock invariants instead become degrees
of freedom for more branch lengths.
\bigskip

\centerline{
\parbox{6.2in}{
\begin{tabular}{c | c c l | c c l | c c l |}
Species& \multicolumn{9}{| c |}{States} \\
\hline
         &  & &2               &    & &3               &    & &4\\
\hline
   2     &  &2&classes         &    & &(same as        &    & &(same as\\ 
         & =&1&for sum         &    & &2 states)       &    & &2 states)\\
         & +&1&branch length   &    & &                &    & &\\
\hline
   3     &  &4&classes         &    &5&classes         &    & &\\
         & =&1&for sum         &   =&1&for sum         &    & &(same as\\
         & +&2&branch lengths  &   +&2&branch lengths  &    & &3 states)\\
         & +&1&clock           &   +&1&clock           &    & &\\
         &  & &                &   +&1&cubic           &    & &\\
\hline
   4     &  &8&classes         &   &14&classes         &   &15&classes\\
         & =&1&for sum         &   =&1&for sum         &   =&1&for sum\\
         & +&3&branch lengths  &   +&3&branch lengths  &   +&3&branch lengths\\
         & +&2&clock           &   +&2&clock           &   +&2&clock\\
         & +&2&Cavender K and L&   +&2&Cavender K and L&   +&2&Cavender K and L\\
         &  & &                &   +&4&cubic           &   +&4&cubic\\
         &  & &                &   +&1&Lake-like       &   +&2&Lake\\
         &  & &                &   +&1&Drolet-Sankoff  &   +&1&Drolet-Sankoff\\
\hline
\end{tabular}
}
}
\newpage

\centerline{FIGURE CAPTION}

\bigskip

Figure 1.  The three-species unrooted tree with the branch lengths shown,
and the three possible kinds of rooted bifurcating trees showing a molecular
clock.

\medskip

Figure 2.  The three different unrooted four-species bifurcating trees.

\medskip

Figure 3.  The two different shapes of rooted bifurcating trees with 4 species
showing a molecular clock.  There are 15 bifurcating tree topologies in all,
each of one or the other of these forms.

\newpage

\centerline{Appendix 1}
\medskip

\centerline{The Three-Species Cubic Invariant}
\medskip

If we let

\begin{equation}
q_i = {3 \over 4} (1-f_i)
\end{equation}

then

\begin{equation}
1-q_i = {1 \over 4} + {3 \over 4} f_i.
\end{equation}

These are in effect the substitution in equation (5) of

\begin{equation}
f  = e^{- 4/3 u t}
\end{equation}

Substituting these for the $q_i$ in equations (7)-(11) we get the equations

\begin{equation}
P_{xxx}  =  {1 \over 16} + {3 \over 16} f_1f_2 + {3 \over 16}f_2f_3 + {3 \over 
16} f_1f_3 + {3 \over 8} f_1f_2f_3
\end{equation}

\begin{equation}
P_{xxy}  =  {3 \over 16} + {9 \over 16} f_1f_2 - {3 \over 16}f_2f_3 - {3 \over 
16} f_1f_3 - {3 \over 8} f_1f_2f_3
\end{equation}

\begin{equation}
P_{xyx}  =  {3 \over 16} - {3 \over 16} f_1f_2 - {3 \over 16}f_2f_3 + {9 \over 
16} f_1f_3 - {3 \over 8} f_1f_2f_3
\end{equation}

\begin{equation}
P_{yxx}  =  {3 \over 16} - {3 \over 16} f_1f_2 + {9 \over 16}f_2f_3 - {3 \over 
16} f_1f_3 - {3 \over 8} f_1f_2f_3
\end{equation}

\begin{equation}
P_{xyz}  =  {6 \over 16} - {6 \over 16} f_1f_2 - {6 \over 16}f_2f_3 - {6 \over 
16} f_1f_3 + {6 \over 8} f_1f_2f_3
\end{equation}
\medskip

Note that there are no linear terms in the $f_i$ in these expressions.  Adding 
the first two of these equations

\begin{equation}
P_{xxx} + P_{xxy}  =  {1 \over 4} + {3 \over 4} f_1f_2
\end{equation}

from which

\begin{equation}
f_1f_2  =  [ 4 (P_{xxx} + P_{xxy}) - 1 ] / 3
\end{equation}

and in analogous fashion from the first and third equations,

\begin{equation}
f_1f_3  =  [ 4 (P_{xxx} + P_{xyx}) - 1 ] / 3
\end{equation}

and from the first and fourth,

\begin{equation}
f_2f_3  =  [ 4 (P_{xxx} + P_{yxx}) - 1 ] / 3.
\end{equation}

Substituting these into the last of the equations, we can eliminate all
the terms $f_1f_2$, $f_1f_3$, and $f_2f_3$, leaving only $f_1f_2f_3$ for which
we then can solve:

\begin{equation}
f_1f_2f_3  =  {2 \over 3} (P_{xxx} - P_{xxy} - P_{xyx} - P_{yxx} + {1 \over 2}) 
\end{equation}

squaring this equation and comparing it to the product of the three
preceding  equations, we find that both have $f_1^2f_2^2f_3^2$ on the
left-hand side, and therefore we can equate the right-hand sides and
get equation (12).


\bigskip

\centerline{Appendix 2}
\medskip

\centerline{Equivalence of Drolet-Sankoff L Invariant and other invariants}
\medskip


Drolet and Sankoff's second quadratic phylogenetic invariant is

\begin{equation}
L_1 = Q_1 Q_2 - Q_3 Q_4 = 0
\end{equation}

where

\begin{equation}
Q_1 = P_{xxxx} + P_{xxyy}
\end{equation}

\begin{equation}
Q_2 = P_{xxxy} + P_{xxyx} + P_{xxyz}
\end{equation}

\begin{equation}
Q_3 = P_{xyxx} + P_{yxxx} + P_{yzxx}
\end{equation}

and

\begin{equation}
Q_4 = P_{xyxy} + P_{xyyx} + P_{xyxz} + P_{xyzx} + P_{yxzx} + P_{yxxz} + 
P_{xyzw}
\end{equation}

We will see that this is not a separate invariant but implies the Cavender
L invariant in the Jukes-Cantor case, given the symmetries of that model,
and the Lake invariants.  The Jukes-Cantor model was the
one Drolet and Sankoff (1990) were considering.

An alternative form of (50) is obtained by noting that since each of the 15
P's occur in exactly one of equations (51) through (54),

\begin{equation}
Q_1 = 1 - Q_2 - Q_3 - Q_4,
\end{equation}

and substituting this into (50) gives 

\begin{equation}
Q_4 = (Q_2 + Q_4)(Q_3 + Q_4).
\end{equation}

We know that in a pattern type like xxyz each of the symbols stands for a
distinct nucleotide.  There are then $4 \times 3 \times 2 = 24$ possible
assignments of nucleotides to the symbols x, y, and z.  In the Jukes-Cantor
case the frequencies of all of these patterns must be equal.  Of these 24
equally frequent patterns, 1/3 have both x and y both pyrimidines or both
purines.  Making a similar argument for the other pattern types,
we find that on classifying the bases into Y and R and those into 0 and 1,
as done above in the discussion of Cavender's K invariant,

\begin{equation}
\begin{array}{c c l}
U_1 & = & S_{0000} + S_{0011}\\
    & = & P_{xxxx} + P_{xxyy} + {1 \over 3} (P_{xxxy} + P_{xxyx}\\
    &   & +P_{xxyz}+P_{xyxx}+P_{yxxx}+P_{yzxx}\\
    &   & +P_{xyxy}+P_{xyyx}+P_{xyzw}) 
\end{array}
\end{equation}

\begin{equation}
\begin{array}{c c l}
U_2 & = & S_{0001} + S_{0010}\\
    & = & {2 \over 3}(P_{xxxy}+P_{xxyx}+P_{xxyz})\\
    &   & + {1 \over 3}(P_{xyxz}+P_{yxxz}+P_{yxzx}+P_{xyzx})
\end{array}
\end{equation}

\begin{equation}
\begin{array}{c c l}
U_3 & = & S_{0100} + S_{0111}\\
    & = & {2 \over 3}(P_{xyxx}+P_{yxxx}+P_{yzxx})\\
    &   &+ {1 \over 3}(P_{xyxz}+P_{yxxz}+P_{yxzx}+P_{xyzx})
\end{array}
\end{equation}

\begin{equation}
\begin{array}{c c l}
U_4 & = & S_{0101} + S_{0110}\\
    & = & {1 \over 3}(P_{xyxz}+P_{yxxz}+P_{yxzx}+P_{xyzx})\\
    &   & + {2 \over 3}(P_{xyxy}+P_{xyyx}+P_{xyzw})
\end{array}
\end{equation}

so that the Cavender L-invariant is

\begin{equation}
U_1 U_4 - U_2 U_3 = 0
\end{equation}

which also can be written as

\begin{equation}
U_4 = (U_2 + U_4) (U_3 + U_4)
\end{equation}

adding equations (58) and (60) gives

\begin{equation}
U_2 + U_4  = {2 \over 3} (Q_2 + Q_4)
\end{equation}

and adding equations (59) and (60) gives

\begin{equation}
U_3 + U_4 = {2 \over 3} (Q_3 + Q_4)
\end{equation}

if we could show that

\begin{equation}
U_4 = {4 \over 9} Q_4
\end{equation}

we would have completed the demonstration that the Drolet-Sankoff L invariant
is a consequence of the Cavender L invariant.  This can be shown, but requires
use of the Lake invariants, equations (24) and (25).  Adding those two
equations shows that the two terms in (60) satisfy:

\begin{equation}
P_{xyxz}+P_{yxxz}+P_{yxzx}+P_{xyzx} = 2 (P_{xyxy}+P_{xyyx}+P_{xyzw})
\end{equation}

If we define

\begin{equation}
V = P_{xyxy}+P_{xyyx}+P_{xyzw}
\end{equation}

then (66) and (60) show that

\begin{equation}
U_4 = {1 \over 3} (2V) + {2 \over 3} V =  {4 \over 3}V
\end{equation}

Since (54) shows that

\begin{equation}
Q_4 = 3V
\end{equation}
% 69

this, together with (68) immediately establishes (65).  Thus the Drolet-
Sankoff L invariant is not autonomous but is a consequence of the
Lake linear invariants, Cavender's L invariant, and the symmetry invariants.
\bigskip

\centerline{Appendix 3}
\medskip

\centerline{Equivalence of some clock invariants and other invariants}
\medskip

For the two tree shapes in Fig. 3, in the four-state case respectively 5 and 
6 clock invariants are given in section 7.  This Appendix derives one of them,
and also shows how all but two of them are equivalent to other invariants.

{\it Equivalence of the two Lake invariants.}  Examining equations (24) and 
(25) and comparing them termwise, equations (30)-(32) can be used to turn
equation (24) into (25), proving that one implies the other.

{\it Equivalence of cubic invariants.}  Equations (29)-(32) were obtained by
noting that any two patterns that can be obtained from each other by
transposing the bases in species 1 and 2 must have the same expected
frequency.  This also immediately establishes that the three-species
marginal pattern frequencies for species 1, 3, and 4 must be the same as
those for species 2, 3, and 4.  The cubic invariants for these two triples of
species are the same functions of those three-species marginal pattern 
frequencies, and hence must be equal.   For the second tree topology in
Fig. 3, the principles are the same, except that species 3 and 4 may
also be transposed (with or without transposing species 1 and 2) without
changing the pattern frequency.  The cubic invariant for species
1, 2, and 3 must then equal that for species 1, 2, and 4.

{\it Clock invariants imply the Cavender K invariant.}  If the $D_{ij}$
are the two-state distances between species $i$ and $j$ (the probabilities
that species $i$ is in a different one of the two states than species $j$), 
section 6a above mentions how it may be shown that Cavender's K invariant is

\begin{equation} % 70
K  =  (D_{14} + D_{23} - D_{24} - D_{13}) - 2 (D_{14} D_{23} - D_{24} D_{13})
\end{equation}

From the equivalence of patterns that have the first two species transposed,
and using equations (20) and (23), we can show that $S_{0101} = S_{0110}$
and that $S_{0100} = S_{0111}$, which leads immediately to

\begin{equation} % 71
   D_{14} = D_{24}.
\end{equation}

In similar fashion it can be shown using (21) and (22) that

\begin{equation} % 72
   D_{13} = D_{23}.
\end{equation}

When these are substituted into equation (70) they immediately establish that
Cavender's K is 0.  Since equations (71) and (72) hold for both of the trees
in Fig. 3, they establish that $K = 0$ for both of them.

{\it Proof of equation (33).}  For the first tree topology in Fig. 3,
in the four-species case, we can imagine dropping species 1 or species 2
from the tree. For the two remaining three-species trees there must be
a clock invariant analogous to equation (13).  The three-species pattern
types can obviously be written in terms of the four-species pattern types.  For
example for the three-species tree that drops species 1,

\begin{equation} % 73
   P_{yxx} = P_{xyxx} + P_{xxyy} + P_{yzxx}
\end{equation}

and there is a similar equation for $P_{xyx}$.  Equation (13) can then be
written in terms of the four-species pattern type frequencies.  Similarly,
for the three-species tree in which species 2 is dropped the same thing
can be done.  The resulting two equations define clock invariants; neither
is equivalent to any of the equations (29)-(32).  This would seem to define
two more clock invariants when we needed only one.  However, it can be
shown that equations (29)-(32) do establish the equivalence of the two
new clock invariants, so that they account for only one degree of freedom
between them.  Adding the two new invariants to express the new clock
invariant in the most symmetrical form, we get equation (33).

\end{document}

JC 4 spp clock:
In tree (((A,B),C),D)

*  xyxx = xyyy
*  xyxy = xyyx
*  xyxz = xyyz
*  xyzx = xyzy

Can easily prove D13=D23, D14=D24,
D24=D34:  xxyx - xxyy + xyxy - xyxx + xyzy - xyzz
D14=D34:  xxyx - xxyy + xyyx - xyyy + xyzx - xyzz   
So to prove D24=D34, using the above, can make it:
    xxyx - xxyy + xyyx - xyyy + xyzx - xyzz

NOTE -- can drop species 2 or species 3 and use the three-species
results above to get

x.yx = y.xx   ( = x.yy )

in fact,
x.yx =  xxyx + xyyx + xyzx
x.yy =  xxyy + xyyy + xyzz

(and from that one can establish  D14 = D24 easily)


and

xy.x = yx.x   ( = xy.y )

It is also true that, dropping species 1

.xyx = .xyy

and dropping species 4

xyx. = xxy.

In fact, can use it to prove that one of Lake's inv's equals the other.
Also that Drolet and Sankoff's first invariant is zero, (their other?)


self-evident for tree (((A,B),(C,D)):
D13 = D23,  D14 = D24, D24 = D23, D14 = D23.


dropping species 1, 2, 3, 4

.xyx = .xxy

x.yx = x.xy

xy.x = xy.y

xyx. = xyy.

From these get the "self evident" equalities
   xxxy = xxyx
*  xyxx = xyyy
*  xyxy = xyyx
*  xyxz = xyyz
*  xyxz = xyzy

