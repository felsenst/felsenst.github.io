\documentstyle{article}
\def\baselinestretch{1.5}
\begin{document}
\begin{flushleft}
{\bf Estimating effective population size from samples of sequences:
inefficiency of pairwise and segregating sites as compared to phylogenetic estimates}
\bigskip

{\bf JOSEPH FELSENSTEIN}
\bigskip

{\it Department of Genetics SK-50, University of Washington, Seattle, Washington 98195}
\bigskip


RUNNING TITLE:  Effective population size from sequence samples
\end{flushleft}

\bigskip
{\bf Summary}

\medskip

\noindent
It is known that under neutral mutation at a known mutation rate a sample
of nucleotide
sequences, within which there is assumed to be no recombination, allows
estimation of the effective size of an isolated population.  This paper
investigates the case of very long sequences, where each pair of sequences
allows a precise estimate of the divergence time of those two gene copies.  The
average divergence
time of all pairs of copies estimates twice the effective population number,
and an estimate can also be derived from the number of segregating sites.
One can alternatively estimate the genealogy of the copies.  This paper shows
how a maximum likelihood estimate of the effective population number can
be derived from such a genealogical tree.  The pairwise and the
segregating sites estimates are
shown to be much less efficient than this maximum likelihood estimate,
and this is verified by computer simulation.  The result implies that
there is much to gain by explicitly taking the tree structure of these
genealogies into account.

\bigskip

\bigskip

{\bf 1. Introduction}

\noindent
The famous paper of Cann, Stoneking, \& Wilson (1987) has focused
attention on the potential of sequence samples from populations to
illuminate population parameters such as effective population sizes and
migration rates.  

We can observe the numbers of substitutions by which sequences differ.
Under the neutral mutation model these differences are expected to accumulate
at a rate of $\mu$ per site per generation.  We can estimate how long ago, in
terms of mutational events, the sequences diverged.  Under genetic
drift, the actual divergence times of the sequences are related to the
effective population size $N_e$.  If $\mu$ is known, we can convert the
mutational scale into a time scale and estimate $N_e$.  If $\mu$ is not
known, the best we can do is to estimate $N_e\mu$.  In this paper I will
discuss the problem in terms of the estimation of $4N_e\mu$.  This is
equivalent to estimation of $N_e$ if $\mu$ is known.

Nei and Tajima (1981) have suggested the use of the average number of
differences per site between two sequences, which they call the {\it nucleotide
diversity}, for estimation of $4N_e\mu$.  Tajima (1983) and Nei (1987)
give a formula for the variance of the estimate.  A slightly different
approach is used by Avise, Ball, and Arnold (1988; see also Avise
1989 and Ball, Neigel, and Avise 1990).  They take pairs of sequences and
make an estimate of divergence
time from each.  They avoid using all pairs of sequences in order to
make the individual estimates more independent.  In an isolated randomly mating
population, we expect the divergence time for a randomly chosen pair of
gene copies to be exponentially distributed with mean $2N_e$.  They fit
the observed distribution of pairwise estimates to an exponential
distribution in order to estimate this quantity.

Watterson (1975) has presented results on the number of segregating sites
at a locus under the neutral ``infinite sites" model, which can also be
used as the basis for an estimate the $N_e$ or $4N_e\mu$.  It is important
to realize when reading the literature on infinite sites models that
the $\mu$ which is described there is the mutation rate {\it per locus};
throughout this paper it will be the mutation rate {\it per site}.

Neither of these estimates makes the most efficient use of such data.  In
this paper I will discuss maximum likelihood estimation, which I will
show is considerably more efficient.  Its efficiency is demonstrated both
theoretically and by computer simulation.  The present paper discusses
only the extreme case of an infinitely long nucleotide sequence;
the more practical matter of dealing efficiently with sequences of
finite length requires computationally intensive techniques that will be
covered elsewhere.  For the moment the objective is simply to show the
weakness of the pairwise and segregating sites approaches.

\bigskip
{\bf 2. A maximum likelihood method}

\medskip
In hopes that it will make efficient use of the data, let us make a
maximum likelihood estimate of $4N_e\mu$ in the case of long sequences.  We
assume that the sequences allow us to estimate their genealogy without
error, and that there is a single such genealogy, i.e., no recombination
has occurred within the sequences during the relevant period of time.
The genealogy is assumed to be produced by Kingman's ``coalescent"
process: we assume that to be a good enough approximation to the
genealogy produced by random genetic drift in a finite population.  This
will be true if $N_e$ is not small.

The results of Kingman (1982a, b) on the coalescent and those of Harding (1971) 
on random trees establish that under the
coalescent the prior distribution on the genealogy assigns equal
probability to all possible bifurcating trees with interior nodes
ordered in time.  The $n$ tips are assumed to be contemporaneous.  Each
interior node has a time, and if two trees differ in the order of
these times they are considered to be different.  Kingman's coalescent
also places a prior on the times.  Starting from the $n$ tips, which occur at
time 0 (the present) the time back to the most recent coalescent
event is exponentially distributed with expectation 4$N_e/[n(n-1)]$.  This is an
approximation, which is excellent for large $N_e$ when $n \ll N_e$, as is
usually the case.

Tavar\'{e} (1984) has reviewed the logic of this approximation.  Strictly
speaking, it requires that as we take larger and larger values of $N_e$ we
observe the process on a time scale whose units are $N_e$ generations.  If we
scale time in expected mutations per site (as we would if we did not know $\mu$),
the mean of the scaled time $u_n$ would be 4$N_e\mu/[n(n-1)]$.  We will
in effect invoke the diffusion approximation, by assuming that 
$N_e \rightarrow \infty$ and $\mu \rightarrow 0$ in such a way that their
product remains constant.  Thus $4N_e\mu$ will equal some constant $\theta$,
and we are approximating the genealogy of the actual population which has
finite values of $N_e$ and $\mu$ by Kingman's coalescent process.

Prior to the most recent coalescence,
uhere were $n-1$ tips, and the interval of scaled time $u_{n-1}$ back to the
previous coalescent event is independently exponentially distributed with mean
$4N_e\mu/[(n-1)(n-2)]$.  In general, if $\theta = 4N_e\mu$ then, scaling time in
expected mutations per site,
\begin{equation} % 1
               u_k  \sim Exponential\; ( \theta/[k(k-1)] )
\end{equation}
with $k = n, n-1, ... 2$.

The topology of the genealogical tree has no information about $\theta$.  We
have already seen that all topologies with time-ordered nodes are
equiprobable, so that their distribution does not depend on $\theta$.  All
information about $\theta$ is contained in the scaled coalescence times and the
intervals $u_k$ between them.

Assume that we have collected a sample of $n$ long sequences from a
random-mating population whose sequences are diverging under neutral
mutation.  The sequences are sufficiently long that we can infer
precisely the genealogical tree connecting those sequences, and from it
the scaled time intervals $u_k$.  Thus we will consider the $u_k$ to have
been observed.

The $k$-th of these has the exponential density function
\begin{equation} % 2
    f_k(u)   =  \frac{k(k-1)}{\theta} \exp [ - \frac{k(k-1)}{\theta} u ]
\end{equation}
so that the full set of $u_k$'s has a joint density function, the
likelihood
\begin{equation} % 3
\begin{array}{l c l}
         L &  =  &  \prod\limits_{k=2}^n f_k(u_k) \\
           &     &  \\
           &  =  &  \prod\limits_{k=2}^n \frac{k(k-1)}{\theta} \exp [- \frac{k(k-1)}{\theta}u_k]
\end{array}
\end{equation}

Taking logarithms, the log-likelihood is

\begin{equation} % 4
\ln L  =  \sum\limits_{k=2}^n \ln k  +  \sum\limits_{k=2}^n \ln (k-1)  -  (n-1) \ln \theta  - \frac{1}{\theta} \sum\limits_{k=2}^n k(k-1) u_k.
\end{equation}

To find the maximum likelihood estimate for $\theta$ we differentiate with
respect to it.  The first two terms, which are logarithms of factorials,
do not contain $\theta$ and disappear, so that

\begin{equation} % 5
\frac{\partial \ln L}{\partial \theta}  =  - \frac{n-1}{\theta} +
 \frac{1}{\theta^2} \sum\limits_{k=2}^n k(k-1) u_k.
\end{equation}

Equating this to zero and solving for $\theta$, we get as the maximum
likelihood estimate

\begin{equation} % 6
\hat{\theta} = \frac{\sum\limits_{k=2}^n k(k-1) u_k}{n-1}.
\end{equation}

This is a simple average of the $k(k-1)u_k$, whose variance is
easily obtained.  Since the variance of the exponential variate $u_k$ is the
square of its mean,

\begin{equation} % 7
\begin{array}{l c l}
Var (\hat{\theta})& = & \frac{1}{(n-1)^2} \sum\limits_{k=2}^n k^2 (k-1)^2
Var(u_k) \\
   &   & \\
  & = &  \frac{1}{(n-1)^2} \sum\limits_{k=2}^n k^2 (k-1)^2 (\theta / [k(k-1)])^2. \\
   &   & \\
  & = &  \frac{\theta^2}{n-1}
\end{array}
\end{equation}

so that the squared coefficient of variation of $\theta$ is simply

\begin{equation} % 8
   \frac{Var (\hat{\theta})}{\theta^2}  =  \frac{1}{n-1}.
\end{equation}

Although the estimate in equation (6) is the maximum likelihood
estimate, it could also be derived by many other methods.  It is the
minimum variance unbiased estimate, the method of moments estimate, and
the weighted least squares estimate as well.

Note that the quantities $k(k-1)u_k$ are independent and all exponentially
distributed with the same expectation $\theta$.  This suggests that it would
be straightforward, given the $u_k$, to construct various goodness-of-fit
tests that could detect 
whether there is a trend in the $u_k$, a trend that would indicate that
the effective population sizes had changed through time.

\bigskip

{\bf 3. The pairwise method}

\medskip
An attractive alternative to maximum likelihood would be to use pairs of
sequences to estimate divergence time, and to average these estimates over
all pairs of tips.  Any two randomly chosen sequences have a time of
divergence which is exponentially distributed with mean $2N_e$, so that
if the divergence time is stated in mutations per site it has mean $2N_e\mu$
which is $\theta/2$.  If we have long sequences, as we assume here, we can
estimate $\theta$ by taking the mean of all these pairwise divergence times and
then estimating $\theta$ by doubling that.  Since the estimate is a mean of
random variables, each of which has expectation $\theta$, it obviously
makes an unbiased estimate of $\theta$.  This method is analogous to the
mean codon difference method of Nei and Tajima (1981) but is not
identical to it: theirs is a pairwise method using mean codon
difference, whereas the present method makes pairwise estimates of
divergence time and then averages them.  Pairwise methods are attractive
because they do not involve estimating the tree topology and have an aura of
robustness.

The aura is, I hope to show, misleading.  To show this, we must compute
the variance of the estimate.  Each pair of sequences has a most recent
common ancestor who occurred at the time of one of the coalescences.  If
$t_k$ is the time (scaled in mutations per site) from the present back until
the coalescent event that reduced $k$ lineages to $k-1$,
then by our earlier definition of the $u_k$,

\begin{equation} % 9
      t_k   =    u_n + u_{n-1} + u_{n-2} + ... + u_k   =   \sum\limits_{i=k}^n u_i
\end{equation}

Figure 1 shows the relationship between the $t_k$ and the $u_k$.
Suppose that we define $m_k$ to be the number of pairs of sequences that
have as their most recent common ancestor the coalescence that occurs when
$k$ lineages are reduced to $k-1$.  Since
every one of the $n(n-1)/2$ pairs of sequences has one or another
coalescence as their most recent common ancestor, it must be true that

\begin{equation} % 10
           \sum\limits_{i=2}^n m_i    =    n (n-1) / 2,
\end{equation}

and we can express the pairwise estimate of $\theta$ in terms of the $m_k$ as

\begin{equation} % 11
         \hat{\theta}_{P}  =  \frac{4 \sum\limits_{i=2}^n m_i t_i}{n(n-1)}.
\end{equation}

The $t_i$ are random variables, but are not independent.  We can use (9) to
express them in terms of the $u_i$, which are independent, obtaining

\begin{equation} % 12
        \hat{\theta}_{P} =  \frac{4 \sum\limits_{i=2}^n m_i 
\sum\limits_{j=i}^n u_j}{n (n-1)}
\end{equation}

which on rearranging summations and changing their limits becomes

\begin{equation} % 13
        \hat{\theta}_{P}  =  \frac{4 \sum\limits_{i=2}^n u_i
\sum\limits_{j=2}^i m_j}{n (n-1)}.
\end{equation}

This is a weighted sum of the $u_i$ but not necessarily a weighted
average.  Let

\begin{equation} % 14
      C_i  =  \sum\limits_{j=2}^i m_j.
\end{equation}

Substituting this into (13), 

\begin{equation} % 15
         \hat{\theta}_{P}  =  \frac{4 \sum\limits_{i=2}^n u_i C_i}{n (n-1)}.
\end{equation}

For each tree topology with ordered internal nodes, we can calculate
the $m_i$, and from those using (14) the $C_i$.   For that tree topology,
we can use (15) to compute the expectation and variance of $\hat{\theta}_P$, using
the fact that the $u_i$ are independently exponentially distributed
according to (1).  The expectation and variance given the $C_i$ are

\begin{equation} % 16
  E[\hat{\theta}_{P} \mid {\bf C}]  =  \frac{4}{n (n-1)} \sum\limits_{k=2}^n
\frac{\theta}{k(k-1)} C_k
\end{equation}

and

\begin{equation} % 17
  Var [\hat{\theta}_{P} \mid {\bf C}]  = \frac{16}{n^2 (n-1)^2} \sum\limits_{k=2}^n
\frac{\theta^2}{k^2 (k-1)^2} C_k^2.
\end{equation}

The expectation (16) will not be the same from one ordered tree topology to
another.  The mean of these means will be $\theta$, but each ordered tree
topology will have a slightly different mean.  For each ordered tree
topology the estimate is biased, but the mean bias is zero.

To complete the calculation from this formula of the variance of the pairwise
estimate of $\theta$, we would need to sum over all ordered tree topologies,
obtaining the
$C_i$ for each, using formulas (16) and (17), and adding the mean of
(17) to the variance of the means (16).  Alternatively we would need a
theory of the $C_i$ so that the summation over ordered tree topologies
would not be necessary.

However, development of such a theory is not necessary, as Tajima (1983)
has developed expressions for the mean and variance of the mean number of
nucleotide differences between pairs of sequences in a sample from a
single population without recombination.  We use the
modification of Tajima's expressions in equations (10.9) and (10.10)
of Nei (1987) which takes the number of sites sampled into account.  As
the number of sites (Nei's $m_{T}$) becomes infinite we have, in our
notation,

\begin{equation} % 18
Var(\hat{\theta}_P) = \frac{2(n^2+n+3)\theta^2}{9n(n-1)}.
\end{equation}

So that from (7) we can compute as the efficiency of the pairwise
method

\begin{equation} % 19
 \frac{Var(\hat{\theta})}{Var(\hat{\theta}_P)} = \frac{9n}{2(n^2+n+3)}.
\end{equation}

Variances of the maximum likelihood and pairwise estimators
and the efficiency of the pairwise estimator, computed from (7), (18), and (19)
are presented in Table 1.  These are also shown as
solid curves in Figures 2 and 3.  It will immediately be apparent that the
efficiency of the pairwise method rapidly becomes small, falling below 0.22
at 20 sequences, 0.11 at 40 sequences, and 0.045 at 100 sequences.
The variances are computed for $\theta = 4$ but as they are proportional
to $\theta^2$ they can be directly computed from this table for any value
of $\theta$ by appropriately multiplying these values.

Note that the variance of the pairwise estimate does not fall to zero,
approaching instead $2\theta^2/9$ as $n \to \infty$.  The reason for
this behavior is that the pairwise estimate takes most of its
information from the times of the earliest few coalescences.  It can be
shown that of all pairs of species, a fraction $(n+1)/(3n-3)$ of them
are expected to be separated by the bottom fork of the genealogical
tree.  This fraction is always greater than $1/3$.  This means that
over $1/3$ of all the information in the pairwise estimate comes from the
time of this one fork!

That this is so is the consequence of a remarkable fact.  If we consider
the two lineages that result from this earliest fork, and wait until a
total of $n$ lineages exist, the distribution of the number of
descendants of the left lineage is uniform on $1, 2, ..., n-1$.  This
follows imediately from Theorem 1 of Harding (1971).  It can also be
obtained by realising that the process of splitting of the left and
right lineages is modelled by Polya's Urn Model.  Let us represent the
two original lineages by balls of different colors.  Splitting a random
lineage corresponds to choosing a random ball, and adding to the urn
another of that color.  Expression (2.3) in Feller (1968) then
establishes that when we reach $n$ balls the fraction that are
of a given color is uniformly distributed.  From this uniform
distribution the expected fraction of pairs of balls that are of different
colors is easily calculated as being $(n+1)/(3n-3)$.  Maddison and Slatkin
(1991) and Slowinski and Guyer (1989) have also used this result in their work
on random trees.
A reviewer has pointed out to me that it also can be obtained directly
from formula (2.3) of Saunders, Tavar\'{e}, and Watterson (1984) by considering
the case $i = n, j = 2$, and $l_1 = l_2 = 2$.  Their formula is then
calculating
the probability that among a population of $N$ organisms reproducing
according to the asexual or haploid version of Moran (1958), when sample
of size $n$ has been traced back to two ancestors, a subsample of a pair
of organisms will still have two distinct parents.  This result is
interesting as in this case the formula holds even when the population size is
finite.

In contrast to the pairwise estimate, the maximum likelihood estimate uses
information
from all coalescence events.  As $n$ increases, it has more and more
coalescences to work from and hence the estimate becomes more and more
accurate.

Ball, Neigel, and Avise (1990) have presented simulation results for
their technique of averaging divergence times for a subset of all pairs of
sequences, chosen so that each sequence is only used once.  Their
results for an average divergence time of 50 pairs of sequences drawn
from a simulated population of 100 sequences shows the expected lack of
bias of the estimator, as well as substantial departures from
independence of the 50 quantities, as expected from the argument given
here. 

\bigskip

{\bf 4. Watterson's method}

\medskip

Watterson (1975) obtained the distributions of the number of segregating
sites for a sample of $n$ sequences from a random-mating population under
an infinite-sites model of mutation.  The infinite-sites model is the
limit of the present model as the mutation rate becomes small, and the
number of sites large.    Although Watterson does not discuss the estimation
of $\theta$ directly, Nei (1987, p. 255) pointed out that an estimate can be
based on the expectation Watterson computed for the number of segregating
sites in the sample, that is, the
number of sites at which there is more than one base present.
Watterson's derivation uses the assumption that
there are finitely-many sites, but assumes that the mutation rate per site $\mu$
is allowed to get small and the population size large at the same rate,
so that their product $N\mu$ remains constant.  Watterson shows that
if $K_n$ is the number of sites showing genetic variation among a sample
of $n$ randomly chosen copies, that the expectation and variance of $K_n$
are to good approximation if $n \ll N$:
\begin{equation} % 20
E(K_n)  =  \theta \sum_{i=1}^{n-1} \frac{1}{i}
\end{equation}
and
\begin{equation} % 21
Var(K_n)  =  E(K_n) + \theta^2 \sum_{i=1}^{n-1} \frac{1}{i^2}
\end{equation}
We will in addition assume that there are a very large number of sites
in the gene, so that although finite, $\theta$ is large.  In this case,
the term $E(K_n)$ makes an unimportant contribution to the right-hand-side
of (21).  An unbiased estimator of $\theta$ is (Nei, 1987, p.255), from (20),
\begin{equation} % 22
\hat{\theta} =  K_n \bigg/ \sum_{i=1}^{n-1} \frac{1}{i}
\end{equation}
and from (21) we can work out the variance of this estimate and compute its
coefficient of variation to be
\begin{equation} % 23
\frac{Var(\hat{\theta})}{\theta^2}  = \frac{ \sum_{i=1}^{n-1} 1/i^2}{\bigl(\sum_{i=1}^{n-1} 1/i\bigr)^2}
\end{equation}
The estimate is unbiased, and as $n \rightarrow \infty$ its coefficient of variation
decreases to zero, so that it makes a consistent estimate.

If the number of sites is not so large, then (23) is increased by the amount
$1/\theta\sum_i (1/i)$, so that (23) is in effect a lower bound on the coefficient
of variation of Watterson's estimate.  We are interested in the cases where
the sequences are very long and hence the number of segregating sites is large.
Thus we are investigating Watterson's method in the cases most favorable to it.

The efficiency of Watterson's estimate must be, taking the ratio of (8) and (23),
no more than
\begin{equation} % 24
\frac{\bigl(\sum_{i=1}^{n-1} 1/i\bigr)^2} {(n-1) \sum_{i=1}^{n-1} 1/i^2}
\end{equation}
The variance (from equation 21) and the efficiency (from 24) of Watterson's
estimate are shown in the rightmost two columns of Table 1.  Actually the
variance shown is the lower bound, using only the second term of the right-hand
side of (21).  This is asymptotically valid, as mentioned above, for large
values of $\theta$.  The value shown is the bound for $\theta = 4$: as with the
other variances in the table, the value will be proportional to $\theta^2$
and this can be used to compute this lower bound for all values of $\theta$,
and hence compute the variance approximately for all large values of $\theta$.

The variance does decrease to zero with increasing $n$, but not as quickly
as does the maximum likelihood estimator.  Efficiency drops with larger $n$,
and although not as low as that for the pairwise estimator, it is
0.42 for 20 sequences, 0.29 for 40, and 0.17 for 100 sequences.


\bigskip

{\bf 5. Simulation results}

\medskip

One might well wonder whether the formula (18) can be applied to the
pairwise estimation method as defined here.  Tajima (1983) and Nei
(1987) derived it as the variance of the average pairwise codon
difference between sequences.  This is not the same as the average
scaled divergence time separating the sequences, which is the quantity of
interest here.  However for an infinitely long sequence, the Tajima-Nei variance
formula is proportional to $\theta^2$.  When sequences are infinitely
long and $\theta$ is small the divergence time will be proportional to the
codon difference between sequences.  Equation (18) will be correct in
that limit.  For a set of sequences the joint distribution of estimated
divergence times will be the same as it is when $\theta$ is small, but
scaled proportional to $\theta$.  Equation (18) should then continue to
apply to the mean of the scaled divergence time estimates.

This argument is sufficiently indirect that it is helpful to check it by
computer simulation.  A large computer simulation has been used both to
compute the expected power of the pairwise method, and to check that the
variances are
correct.  Two programs were written in MIPS Pascal on a Digital DECstation
3100 and a DECstation 5000.  The first program is given the number of
sequences to be
sampled, and the number of replicates, as well as a random number seed.
It simulates the coalescent, starting with a number of lineages equal to
the desired number of sequences, drawing pairs of lineages and the sampling
the times of their immediate common ancestor from the distribution (1).
This technique of simulating the coalescent by working backwards was
pioneered by Hudson (1983).  The true genealogical tree is recorded,
including the ordered tree topology as well as the actual times of the
interior nodes.  

From the time-ordered tree topology we can compute the quantities $m_i$
which are used in (14), (16), and (17).  This gives us the expectation
and variance of the estimate of $\theta$ for that ordered tree topology.
The overall expectation of the estimate of $\theta$ will be the average of
(16) over all tree topologies.  The variance of the estimate will be
the average of the within-topology variances (17), plus the variance of
the expectations (16), averaging over all ordered tree topologies.  Lacking
a theory of the statistical behavior of the $C_i$ we cannot compute the
expectation and variance of the estimate of $\theta$ in the pairwise method.

The approach taken here has been to compute these approximately by
sampling a large number of ordered tree topologies. The second computer
program takes each one and computes (16) and (17).  These then can be
used to compute the approximate expectation and variance of the estimate
of $\theta$ under the pairwise method.  Table 2 shows the results.  Its
penultimate column shows 
the variance of the pairwise estimate in the same case, as determined
by this combination of simulation and theory.  The computed expectations
of the estimate of $\theta$ are not shown; they were always quite close to
$\theta$ and support the conclusion that the pairwise method, like the maximum
likelihood method, is unbiased.  The variances of the pairwise estimate
computed from the simulation are quite close to the values obtained from
the Tajima-Nei formula.

The computer programs that simulate the coalescent record not only the
ordered tree topology, but the true ages of the interior nodes as well.  This
makes it possible in each case, using (6), (11), and (22), to compute what the
maximum likelihood, the pairwise, and the segregating sites estimates of $\theta$ would be for that
tree, given that long enough sequences were available to estimate the
genealogical tree exactly.  So for each simulated tree we get three estimates
of $\theta$.  The means of these estimates were in accord with the
unbiasedness of the estimates and are not shown here.  The first six columns of
Table 2 show the
empirical variances of the three estimates, and the ratios of variances,
which are estimates of the efficiency of the pairwise and the segregating sites
methods, computed
directly from the simulations without using equations (8), (16), 
(17), or (24).  Figures 2 and 3 show the variances and efficiencies from both
tables.  The
lines connect the theoretical variances and efficiencies from
Table 1, and the points are the empirical variances and efficiencies from
the simulation results in Table 2.  Again, the results show
excellent agreement of the theory with the
simulations.
\bigskip

{\bf 6. Relation to infinite-sites models}
\medskip

The present approach may at first seem unrelated to the papers by
Strobeck (1983), Ethier and Griffiths (1987), and Griffiths (1989) which
calculate probabilities of
different kinds of samples that might be taken from a population undergoing
an infinite-sites model of mutation, in the absence of recombination.
Strobeck (1983) used a diffusion approximation to derive recurrence
relations among the probabilities of the different possible kinds of
observed samples for two or three sequences.  Ethier and Griffiths (1987)
gave a general recursion formula for these probabilities for any number
of sequnces.  Strobeck (1983) shows how to use these formulas to make maximum
likelihood estimates of $\theta$.  Griffiths (1989)
describes a computer program that can calculate these probabilities.  The
probability of the observed sample is of course the likelihood, and by
varying $\theta$ one can compute the likelihood curve and the maximum
likelihood estimate.

The effect of having infinitely long sequences, as assumed here, is most
easily seen by considering Strobeck's
equations, for the case where two sequences are observed.
As we use $\mu$ for the mutation rate at a single site, suppose that
$U$ is the mutation rate for the whole locus, and that $\Omega = 4N_eU$.
This is the quantity that these authors call $\theta$ in their equations.
Strobeck's equation (2) for the case of two different sequences shows,
for $n = 2$ and $m_1 = m_2 = 1$, that the distribution of the number of
mutations by which two sequences differ is geometric, with mean
$\Omega$.  The variance of this distribution will be $\Omega^2 + \Omega$.
The maximum likelihood estimate of $\Omega$ turns out to be simply the observed
number of mutations by which the sequences differ, so that the mean and variance
of the estimate are also $\Omega$ and $\Omega^2 + \Omega$.  The squared
coefficient of variation is the $1 + 1/\Omega$.

This exceeds the value calculated in this paper by $1/\Omega$.  The extra
variation is due to the inaccuracy of estimating the tree (which in this
case is simply the divergence time of the two sequences).  As the number
of sites is taken larger and larger for a given value of $\theta$,
$\Omega \rightarrow \infty$ so that $1 + 1/\Omega \rightarrow 1$.
Thus the extra variability of the estimate due to the finiteness of
$\Omega$ disappears, and we are left only with the variability due to the
randomness of the true genealogy, the randomness accounted for in this paper.
I expect that the same behavior will occur when more sequences are
considered, and that this could be verified by a detailed consideration of
the equations in Strobeck (1983) and Griffiths (1989).  If so, there would
be no conflict between their results and mine.
\bigskip

{\bf 7. Limitations}
\medskip

The proofs above rely on a number of assumptions that are questionable:

{\it (i) No recombination.}  It is assumed that the sequences have a
genealogy which is a branching tree, and this can only happen when
there is no recombination in the region in any of the lineages leading
back to the common ancestor sequence.  Recombination would result in a
single sequence (the recombinant) having contributions from two or
more ancestors.  With a single recombination, the front and rear ends
of the sequences would have different, but similar, genealogical trees.
Treating such cases is a major challenge for the future.  For
mitochondrial DNA sequences, strict maternal inheritance guarantees that
this problem does not arise.

{\it (ii) Infinitely long sequences.}  The analysis here was enormously
facilitated by the assumption that we have infinitely long sequences
so that they allow us to estimate the details of the genealogy without
error.  The
statistical error in this study is thus only the error that comes from
having a finite number of sequences.  If instead we had sequences of
finite length, as we always would, the maximum likelihood method becomes
much more difficult computationally.  I hope to present in a separate
paper a computationally intensive procedure that can make a maximum
likelihood estimate of effective population size from samples of
finite-length sequences.  In that case there is additional statistical
error from the imprecision of estimation of both the topology of the
genealogy and the divergence times.  It would
inflate the error of all of the estimates.
It is not obvious which one would be affected most, but it is at least
possible that, as both the numerator and the denominator of the
efficiencies are affected, that the efficiency of pairwise and segregating
sites methods would not be as low when the sequence lengths were small.
However when
sequences are long, the variances and efficiencies must approach the values
given here.

{\it (iii) Lack of geographical subdivision.}  It has been assumed that there
is only one population, mating at random.  If there are a number of
local populations exchanging migrants, the notion of effective
population size becomes complicated: we have both local effective sizes
(Sewall Wright's (1940) concept of ``neighborhood size") and an
effective number for the whole species which can be considerably larger.
Both contribute to the rate of coalescence of lineages.  When two
lineages are in the same local population, it will be possible for them
to coalesce in the previous generation, while when they are in different
local populations they cannot.  Slatkin (1987), Takahata (1988) and
Takahata and Slatkin (1990)
have investigated this for two populations exchanging
migrants.  The distribution of the time to coalescence of two lineages
collected from the same local population
is no longer exponential and is not easy to obtain.  Slatkin and
Maddison (1989) have proposed an estimate of migration rate between
the populations for the case of infinitely long sequences.  For the case
where the sequences are not very divergent their estimate is probably
close to being a maximum likelihood estimate.

It should be obvious that much remains to be done; it may be doubted
whether the methods of analysis will ever catch up with the collection
of data.

\bigskip

\medskip

{\small I am grateful to Monty Slatkin for frequent discussions of the
coalescent, for pointing out the relevance of Polya's urn model, and for
comments on the manuscript of this paper.
I also thank Charles Geyer for comments on the lack of consistency of
the pairwise method, David Aldous for comments on the logical status
of the maximum likelihood estimator, and John Wakeley and Avigdor Beiles for
finding
typographical errors.   I wish also to thank Richard Hudson, associate editor
of {\it Genetics}, and his reviewers for helpful comments on an earlier
version of this paper, in particular for pointing out the relevance of the
work on infinite-sites models and suggesting notational reforms.  This work
was supported by National Science
Foundation grants numbers BSR-8614807 and BSR-8918333, and by National Institutes of Health grant number 1 R01 GM 41716-01.}

\bigskip

{\bf References}
\medskip
{
\setlength{\parindent}{-0.2in}

Avise, J. C. (1989)  Gene trees and organismal histories: a phylogenetic
approach to population biology.  {\it Evolution}  {\bf 43}, 1192-1208.
\medskip

Avise, J. C., Ball, R. M., Jr. \& Arnold, J. (1988)  Current versus
historical population sizes in vertebrate species with high gene flow: a
comparison based on mitochondrial DNA polymorphism and inbreeding theory
for neutral mutations.  {\it Molecular Biology and Evolution}  {\bf 5}, 331-344.  
\medskip

Ball, R. M., Jr., Neigel, J. E. \& Avise, J. C. (1990)  Gene genealogies
within the organismal pedigrees of random-mating populations.  {\it Evolution}
{\bf 44}, 360-370.
\medskip

Cann, R. L., Stoneking, M. \& Wilson, A. C.  (1987)  Mitochondrial DNA and
human evolution.  {\it Nature}  {\bf 325}, 31-36.
\medskip

Ethier, S. N. \& Griffiths, R. C.  (1987)  The infinitely-many-sites model as
a measure-valued diffusion.  {\it Annals of Probability} {\bf 15}, 515-545.
\medskip

Feller, W.  (1968)  {\it An Introduction to Probability Theory and Its
Applications}, 3rd edn.  John Wiley, New York.
\medskip

Griffiths, R. C.  (1989)  Genealogical tree probabilities in the
infinitely-many-site model.  {\it Journal of Mathematical Biology} {\bf 27},
667-680.
\medskip

Harding, E. F.  (1971)   The probabilities of rooted tree shapes
generated by random bifurcation.  {\it Advances in Applied Probability}
{\bf 3}, 44-77.
\medskip

Hudson, R. R. (1983)  Testing the constant-rate neutral allele model
with protein sequence data.  {\it Evolution}  {\bf 37}, 203-217.
\medskip

Kingman, J. F. C. (1982a)  The coalescent.  {\it Stochastic Processes and Their
Applications}  {\bf 13}, 235-248.
\medskip

Kingman, J. F. C. (1982b)  On the genealogy of large populations.  {\it
Journal of Applied Probability}  {\bf 19A}, 27-43.
\medskip

Maddison, W. P. \& Slatkin, M. (1991)  Null models for the number
of evolutionary steps in a character on a phylogenetic tree.  {\it Evolution}
{\bf 45}, 1184-1197.
\medskip

Moran, P. A. P.  (1958)  Random processes in genetics.  {\it Proc. Camb. Phil.
Soc.} {\bf 54}, 60-71.
\medskip

Nei, M. \&  Tajima, F.  (1981)   DNA polymorphism detectable by
restriction endonucleases.  {\it Genetics} {97}, 145-163.
\medskip

Nei, M. (1987)   {\it Molecular Evolutionary Genetics}.  Columbia
University Press, New York.
\medskip

Saunders, I. W., Tavar\'{e}, \& Watterson, G. A. (1984)  On the genealogy of nested subsamples from a
haploid population.  {\it Advances in Applied Probability}  {\bf 16}, 471-491.
\medskip

Slatkin, M. (1987)  The average number of sites separating DNA sequences
drawn from a subdivided population.  {\it Theoretical Population Biology}  {\bf 32}, 42-49.
\medskip

Slatkin, M. (1989)  Detecting small amounts of gene flow from phylogenies of
alleles.  {\it Genetics}  {\bf 121}, 609-612.
\medskip

Slatkin, M. \& Maddison, W. P.  (1989)  Cladistic measure of gene flow
inferred from the phylogenies of alleles.  {\it Genetics}  {\bf 123}, 603-613.
\medskip

Slowinski, J. G. \& Guyer, C. (1989) Testing the stochasticity of patterns
of organismal diversity: an improved null model.  {American Naturalist} {\bf 134}, 907-921.

Tajima, F. (1983)   Evolutionary relationship of DNA sequences in finite
populations. {\it Genetics} {\bf 105}, 437-460.
\medskip

Takahata, N. (1988)  The coalescent in two partially isolated diffusion
populations.  {\it Genetical Research}  {\bf 52}, 213-222.
\medskip

Takahata, N. \& Slatkin, M. (1990)  Genealogy of neutral genes in two partially
isolated populations.  {\it Theoretical Population Biology} {\bf 38}, 331-350.
\medskip

Tavar\'{e}, S.  (1984)  Line-of-descent and genealogical processes, and their
applications in population genetics models.  {\it Theoretical Population
Biology}  {\bf 26}, 119-164.
\medskip

Watterson, G. A.  (1975)  On the number of segregating sites in genetical
models without recombination.  {\it Theoretical Population Biology} {\bf 7}, 256-276.
\medskip

Wright, S. (1940)  Breeding structure of populations in relation to
speciation.  {\it American Naturalist}  {\bf 74}, 232-248.
\medskip
}

\newpage

Table 1.  Theoretical variance of the maximum likelihood estimate of
$\theta$, when $\theta = 4$, of the pairwise method, using equations
(7) and (18), and of Watterson's method, from equation
(23).  The efficiencies of the pairwise method, from
equation (19) and of Watterson's method, from equation (24), are also
shown.

\bigskip
\begin{tabular}{r c c c c c}
\medskip             
$n$ &  $Var(ML)$ &  $Var(P)$ & $Efficiency(P)$ & $Var(W)$ & $Efficiency(W)$ \\
& & & & & \\
  2 &   16.0000 &   16.0000 &   1.00000 &   16.0000 &   1.00000 \\
  3 &    8.0000 &    8.8889 &   0.90000 &    8.8889 &   0.90000 \\
  4 &    5.3333 &    6.8148 &   0.78261 &    6.4793 &   0.82313 \\
  5 &    4.0000 &    5.8667 &   0.68182 &    5.2480 &   0.76220 \\
  6 &    3.2000 &    5.3333 &   0.60000 &    4.4917 &   0.71243 \\
  7 &    2.6667 &    4.9947 &   0.53390 &    3.9754 &   0.67080 \\
  8 &    2.2857 &    4.7619 &   0.48000 &    3.5980 &   0.63528 \\
  9 &    2.0000 &    4.5926 &   0.43548 &    3.3085 &   0.60451 \\
 10 &    1.7778 &    4.4642 &   0.39823 &    3.0784 &   0.57751 \\
 15 &    1.1429 &    4.1143 &   0.27778 &    2.3850 &   0.47918 \\
 20 &    0.8421 &    3.9579 &   0.21277 &    2.0259 &   0.41567 \\
 25 &    0.6667 &    3.8696 &   0.17228 &    1.8001 &   0.37034 \\
 30 &    0.5517 &    3.8130 &   0.14469 &    1.6424 &   0.33593 \\
 35 &    0.4706 &    3.7737 &   0.12470 &    1.5245 &   0.30868 \\
 40 &    0.4103 &    3.7447 &   0.10956 &    1.4323 &   0.28643 \\
 45 &    0.3636 &    3.7226 &   0.09768 &    1.3577 &   0.26784 \\
 50 &    0.3265 &    3.7050 &   0.08813 &    1.2957 &   0.25201 \\
 60 &    0.2712 &    3.6791 &   0.07371 &    1.1980 &   0.22637 \\
 70 &    0.2319 &    3.6608 &   0.06334 &    1.1236 &   0.20637 \\
 80 &    0.2025 &    3.6473 &   0.05553 &    1.0646 &   0.19024 \\
 90 &    0.1798 &    3.6368 &   0.04943 &    1.0163 &   0.17688 \\
100 &    0.1616 &    3.6285 &   0.04454 &    0.9759 &   0.16561 \\
150 &    0.1074 &    3.6038 &   0.02980 &    0.8405 &   0.12776 \\
200 &    0.0804 &    3.5916 &   0.02239 &    0.7607 &   0.10569 \\
300 &    0.0535 &    3.5795 &   0.01495 &    0.6661 &   0.08033 \\
400 &    0.0401 &    3.5734 &   0.01122 &    0.6093 &   0.06582 \\
500 &    0.0321 &    3.5698 &   0.00898 &    0.5700 &   0.05625 
\end{tabular}
\newpage

Table 2.  Empirical variances of the coalescent, pairwise, and Watterson
estimates
of $\theta$ in the computer simulations, plus the empirical efficiency
of the pairwise and Watterson method obtained by taking the ratio of
the coalescent variance to those for these other two.  The seventh column
shows the value of the variance of the
pairwise method obtained by simulation, computing within- and
between tree topology variances by sampling ordered tree topologies for
this value of $\theta$ and using equations (16) and (17), for different
values of $n$.  The eighth is
the number of ordered tree topologies sampled in the simulation.\\
\begin{tabular}{r c c c c c c r}
\medskip
$n$ &  $Var(ML)$ &  $Var(P)$ & $Eff(P)$ & $Var(W)$ & $ Eff(W)$ & $Var'(P)$ & $Replicates$ \\
   2 & 16.0510 & 16.0510 & 1.00000 & 16.0510 & 1.00000 & 16.0000 &  1,000,000 \\
   3 &  7.9562 &  8.8401 & 0.90001 &  8.8401 & 0.90001 &  8.9709 &  1,000,000 \\
   4 &  5.3259 &  6.8132 & 0.78170 &  6.4732 & 0.82277 &  6.7988 &  1,000,000 \\
   5 &  3.9900 &  5.8340 & 0.68391 &  5.2227 & 0.76397 &  5.8855 &  1,000,000 \\
   6 &  3.1971 &  5.3470 & 0.59793 &  4.5022 & 0.71012 &  5.3343 &  1,000,000 \\
   7 &  2.6553 &  4.9679 & 0.53449 &  3.9524 & 0.67181 &  4.9923 &  1,000,000 \\
   8 &  2.2895 &  4.7656 & 0.48042 &  3.6010 & 0.63579 &  4.7590 &  1,000,000 \\
   9 &  2.0020 &  4.5827 & 0.43686 &  3.2978 & 0.60707 &  4.5990 &  1,000,000 \\
  10 &  1.7770 &  4.4475 & 0.39954 &  3.0746 & 0.57796 &  4.4661 &  1,000,000 \\
  15 &  1.1433 &  4.1089 & 0.27824 &  2.3835 & 0.47966 &  4.1124 &  1,000,000 \\
  20 &  0.8446 &  3.9564 & 0.21346 &  2.0267 & 0.41671 &  3.9557 &  1,000,000 \\
  25 &  0.6648 &  3.8759 & 0.17153 &  1.8042 & 0.36848 &  3.8694 &  1,000,000 \\
  30 &  0.5524 &  3.8092 & 0.14501 &  1.6427 & 0.33625 &  3.8144 &  1,000,000 \\
  35 &  0.4704 &  3.7637 & 0.12499 &  1.5194 & 0.30963 &  3.7717 &  1,000,000 \\
  40 &  0.4098 &  3.7458 & 0.10940 &  1.4330 & 0.28596 &  3.7440 &  1,000,000 \\
  45 &  0.3636 &  3.7244 & 0.09762 &  1.3579 & 0.26775 &  3.7229 &  1,000,000 \\
  50 &  0.3257 &  3.6870 & 0.08833 &  1.2933 & 0.25183 &  3.7036 &  1,000,000 \\
  60 &  0.2707 &  3.6853 & 0.07345 &  1.2014 & 0.22530 &  3.6782 &   500,000 \\
  70 &  0.2314 &  3.6809 & 0.06287 &  1.1275 & 0.20526 &  3.6627 &   500,000 \\
  80 &  0.2019 &  3.6484 & 0.05534 &  1.0631 & 0.18993 &  3.6498 &   500,000 \\
  90 &  0.1794 &  3.6164 & 0.04960 &  1.0116 & 0.17731 &  3.6354 &   500,000 \\
 100 &  0.1614 &  3.6249 & 0.04451 &  0.9723 & 0.16594 &  3.6281 &   500,000 \\
 150 &  0.1073 &  3.5991 & 0.02983 &  0.8383 & 0.12805 &  3.6063 &   500,000 \\
 200 &  0.0803 &  3.6124 & 0.02222 &  0.7626 & 0.10526 &  3.5922 &   500,000 \\
 300 &  0.0536 &  3.5718 & 0.01500 &  0.6677 & 0.08022 &  3.5794 &   200,000 \\
 400 &  0.0400 &  3.5772 & 0.01119 &  0.6132 & 0.06527 &  3.5729 &   200,000 \\
 500 &  0.0325 &  3.8180 & 0.00852 &  0.5953 & 0.05462 &  3.5725 &    10,000 
\end{tabular}
\newpage

\centerline{FIGURE CAPTIONS}

Figure 1.  A genealogical tree, showing the relationship between the
$u_i$ and the $t_i$.  Both are measured in generations back from the
present.

\bigskip

Figure 2.  The variances of the estimates of $\theta$ from the simulation
from n=3 to n=500 when computed by the coalescent
maximum likelihood method (circles) the pairwise method 
(squares), and Watterson's method (pluses).  The continuous curves are the
corresponding theoretical values from
equations (7), (18), (23).  The value for n=500 is based on many fewer
simulations than the other values.

\bigskip

Figure 3.  Theoretical and empirical values of the efficiency of
the pairwise and segregating sites estimates of $\theta$
from n=2 to n=500.  The lower curve shows the theoretical value,
computed from equation (19), and the upper curve the theoretical efficiency
of Watterson's method, from (24).  Squares show the empirical values obtained
by taking the ratio of the empirical variances among replicates of the
coalescent and pairwise estimates, and pluses the empirical values for
Watterson's method. 

\end{document}
