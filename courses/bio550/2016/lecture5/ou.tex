\documentclass[pdf,bluish,slideColor,colorBG]{prosper}
\hypersetup{pdfpagemode=FullScreen}
\usepackage{color}
\usepackage{graphicx}
\usepackage{amsfonts}
\usepackage{amsmath}
\def\baselinestretch{0.7}
\parindent 0.3in
\hyphenpenalty=10000
\tolerance=10000
\pagestyle{empty}

\def\Prob{{\rm Prob\;}}
\def\prob{{\rm \;Prob\;}}
\def\Var{{\rm Var}}        % Var
\def\Cov{{\rm Cov}}        % Cov

\DeclareSymbolFont{AMSb}{U}{msb}{m}{n}
\DeclareMathSymbol{\expect}{\mathalpha}{AMSb}{'105}

% bold math (use \bm{...})
\def\bm#1{\mathpalette\bmstyle{#1}}
\def\bmstyle#1#2{\mbox{\boldmath$#1#2$}}

\title{Optimum selection and OU processes}

\author{Joe Felsenstein}

\institution{Biology 550D}

\subtitle{\small \\ 22 November 2016}


\definecolor{orange}{rgb}{1.0,0.8,0.0}
\definecolor{Dandelion}{rgb}{0.8,0.4,0.3}
\definecolor{golden}{rgb}{1.0,0.75,0.2}
%\definecolor{golden}{rgb}{1.0,0.8,0.3}
\definecolor{purple}{rgb}{0.6,0.2,0.6}
\definecolor{darkblue}{rgb}{0.1,0.1,0.6}
\definecolor{yellow}{rgb}{1.0,1.0,0.0}
\definecolor{brightred}{rgb}{1.0,0.,0.0}
\definecolor{black}{rgb}{0.0,0.0,0.0}
\definecolor{white}{rgb}{1.0,1.0,1.0}
\definecolor{purple}{rgb}{0.8,0.0,0.8}

% sets backgroundcolor for whole document 
%\pagecolor{darkblue}
%\pagecolor{white}
% sets text color
%\color{yellow}
%\color{black}
% to change just a few words
% using \textcolor{color}{text}

\DeclareSymbolFont{AMSb}{U}{msb}{m}{n}
\DeclareMathSymbol{\expect}{\mathalpha}{AMSb}{'105}

\def\Prob{{\rm Prob\;}}
\def\prob{{\rm \;Prob\;}}
\def\Var{{\rm Var}}        % Var
\def\Cov{{\rm Cov}}        % Cov

\begin{document}

% Source of covariation (VA, selective)
% 
% Ornstein-Uhlenbeck (Marguerite?)
% Chasing a peak
% Go where peak goes
% Smoothing caused by this
% Equilibrium covariances
% Tree covariation
% 


\maketitle

{
\parindent=0in

\overlays{11}{
\begin{slide}[Replace]{Analyzing various kinds of data with morphometrics}

A casual outline. Participation compulsory. Issues for discussion:

\begin{itemstep}
\item Within individuals. Growth: What about allometry?
\begin{enumerate}
\item Sequence of observations of individuals
\item Observations of different individuals of various sizes
\end{enumerate}
\item Multiple individuals within a population
\begin{enumerate}
\item Phenotypic variation
\item Quantitative genetic breeding designs
\item Pedigrees of individuals
\end{enumerate}
\item Multiple populations
\begin{enumerate}
\item Are populations different.  Which ones?
\item Are differences correlated with the environment?
\end{enumerate}
\item Multiple species on a phylogeny
\end{itemstep}

\end{slide}
}

\begin{slide}[Replace]{What causes change in quantitative characters? }
\bigskip

For neutral mutation and genetic drift, can show that for a quantitative
character with additive genetic variance $~\mathsf{V_A}~$ and population size $\mathsf{~N~}$
the genetic (additive) value of the population mean is:
\bigskip

\[
\mathsf{\Var (\Delta \bar{g}) \ = \ V_A / N}
\]
\bigskip

If mutation and drift are at equilibrium:
\[
\mathsf{\expect\left[V_A^{(t+1)}\right] \ = \ V_A^{(t)} \left(1 - \frac{1}{2N}\right) + V_M}
\]
\bigskip

\end{slide}

\begin{slide}[Replace]{In neutral traits additive genetic variance rules}
\bigskip

so that
\[
\mathsf{\expect\left[V_A\right] \ = \ 2N V_M}
\]

whereby
\[
\mathsf{\Var[\Delta {\bar g}] \ = \ \left(2N V_M \right) / N  \ = \  2 V_M}
\]

an analog of Kimura's result for neutral mutation.
\bigskip

Thus to transform characters to independent Brownian motions of equal
evolutionary variance, we could use the additive genetic variance $\mathsf{V_A}$.

\end{slide}

\begin{slide}[Replace]{With multiple characters ... }
\bigskip

There is a precise analogue of this for multiple characters:

\[
\mathsf{\expect\left[{\bf A}^{(t+1)}\right] \ = \ {\bf A}^{(t)} \left(1 -
\frac{1}{2N}\right) + {\bf M}}
\]
\noindent
where $\mathsf{~{\bf A}~}$ is the additive genetic covariances, and
$\mathsf{~{\bf M}~}$ is the covariance matrix of pleiotropic effects of
mutation.

\[
\mathsf{\expect\left[{\bf A}\right] \ = \ 2N\, {\bf M}}
\]

\noindent
and

\[
\mathsf{\Var[{\bf \Delta {\bar g}}] \ = \ \left(2N {\bf M} \right) / N  \ = \
2 {\bf M}}
\]
\bigskip

\noindent
so as long as mutations cause expected change zero (i.e. they are not near some
biological limit), the effect of genetic drift is that the mean phenotype
wanders according to the mutational covariances.  The constant additive genetic variance assumption was used by Russ Lande.

\end{slide}

\begin{slide}[Replace]{References for multivariate Brownian motion}

Felsenstein, J.  2004.  {\it Inferring Phylogenies.}  Sinauer Associates,
Sunderland, Massachusetts. \textcolor{purple}{\bf [See particularly
chapters 23, 24, 25]}
\medskip

Felsenstein, J. 1988. Phylogenies and quantitative characters. {\it Annual
Review of Ecology and Systematics} {\bf 19:} 445-471. \textcolor{purple}{\bf
[Review with selection as well as neutral mutation]}
\medskip

Felsenstein, J. 2002. Quantitative characters, phylogenies, and
morphometrics.pp. 27-44 in {\it Morphology, Shape, and Phylogenetics}, ed.
N. MacLeod. Systematics Association Special Volume Series 64. Taylor
and Francis, London.  \textcolor{purple}{\bf[Review repeating 1988 material and going into some
more detail on the question of threshold models.]}
\medskip

Lynch, M. and W. G. Hill. 1986. Phenotypic evolution by neutral mutation.
{\it Evolution} {\bf 40:} 915-935.
\medskip

\end{slide}

\end{document}

