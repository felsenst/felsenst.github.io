\documentclass[12pt]{article}
\def\baselinestretch{1.0}
\setlength{\topmargin}{0pt}
\setlength{\textheight}{640pt}
\setlength{\oddsidemargin}{0pt}
\setlength{\evensidemargin}{60pt}
\setlength{\textwidth}{477pt}
\setlength{\footskip}{30pt}
\parindent 0.3in
\hyphenpenalty=10000
\tolerance=10000
\pagestyle{empty}

\begin{document}
{\large
\noindent
Genome 570 \hfill Winter, 2016\\
Phylogenetic Inference \hfill J. Felsenstein\\
}

\vfill

\centerline{Homework no. 3}
\medskip

\centerline{Due Friday, February 19}
\bigskip

\begin{itemize}
\item Using the same nucleotide or protein sequence data set that you used
for Homework \#2, infer a maximum likelihood phylogeny.  (If that data set
turned out to have big problems, you may choose another one).
\item If you have been using protein sequences you should use a program that
has a 20$\times$20 amino acid model such as Jones-Taylor-Thornton (JTT) or
other some reasonable one like WAG.  For DNA sequences use one of the usual
DNA models.
\item The ML analysis should include allowing gamma-distributed variation
of rates from site to site.  If you need to choose the parameter value
for the amount of rate variation, do so by maximum likelihood.  That is,
either have the program optimize the parameter (if it can) or do so yourself
by trying different values of the parameter and finding the one that leads
to the highest likelihood.  The parameter will either be the Gamma
distribution's ``shape parameter'' $\alpha$ or its Coefficient of Variation
which is $1/\sqrt{\alpha}$.
\item Since for some programs, 
trying different values of the rate variation parameter may be
very slow, one way that works almost as well would be to get a tree topology
for one value of the parameter, and then feed it in as a user-defined tree
so that the program only uses that topology but does infer new branch lengths
for it.  Then try different parameter values in different runs, which should be
much quicker than doing a full search, and find the parameter value
with the highest likelihood.
\item If we then start with that parameter value and do a search among tree
topologies does the topology found differ?  If so, and we then use that
new topology to again re-estimate the parameter, Does this process converge?
\item Report also on which sites show the smallest rate of change, and
which show the most, and whether that pattern makes biological sense (third 
positions?  active sites?).
\item Also (after the rate variation parameter is determined) use it
to do a bootstrap analysis with at least 100 bootstrap replicates.
The bootstrap analysis includes making a majority-rule consensus
tree of the 100 (or more) results.
\item Some program packages do not do ML.  I am not insisting that you use
PHYLIP.  In fact if you used it before, I would suggest trying another
package, to broaden your experience.   You can find packages in my Phylogeny 
Programs list (use that
phrase in a search engine to find the pages).  PAUP*, Phyml, RAxML, and MEGA
are popular choices.
\item Make sure in your report to show the ML estimate of the tree, and
report on the rate variation parameters, which sites seem to have high
and low rates (not so much the detailed site numbers as the description of
which functional regions in the molecule have high and low rates).  For the
bootstrap analysis report the bootstrap P values.   Does the amount of
bootstrap support make sense in terms of which parts of the tree seem
biologically sensible?
\item Comment on the programs you used, how well they worked and how easy they
were to use.
\end{itemize}
\bigskip

\noindent
e-mail me {\tt (joe (at) gs.washington.edu)} with a report in PDF or MS Word
(.docx) form on the results.   Mercy will once again be granted if you need
more time, but a new assignment is coming on February 26 which {\it must} be 
turned in a week later as otherwise I will out of time in the quarter to grade
them..
\vfill

\vfill

\end{document}

