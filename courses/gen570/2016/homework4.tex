\documentclass[12pt]{article}
\usepackage{hyperref}
\def\baselinestretch{1.0}
\setlength{\topmargin}{-30pt}
\setlength{\textheight}{660pt}
\setlength{\oddsidemargin}{0pt}
\setlength{\evensidemargin}{60pt}
\setlength{\textwidth}{477pt}
\setlength{\footskip}{30pt}
\parindent 0.3in
\hyphenpenalty=10000
\tolerance=10000
\pagestyle{empty}

\begin{document}
{\large
\noindent
Genome 570 \hfill Winter, 2016\\
Phylogenetic Inference \hfill J. Felsenstein\\
}

\vfill

\centerline{Homework no. 4}
\medskip

\centerline{Due Wednesday, March 2}
\bigskip

\noindent
Take the same data set you used for the previous homework (if
possible) and do Bayesian inference of the phylogeny.
Here are eight programs for this (some work only on DNA, only on RNA,
or only on protein sequences):  When this PDF is viewed on-line,
the links below should be active and lead you to the programs' web sites.

\hspace*{0in}\hspace{-0.3in}{\parindent=-0.75in
\fbox{
\begin{tabular}{l l}
{\tt MrBayes} & {\tt \url{http://www.mrbayes.net}}\\
{\tt BEAST} & {\tt \url{http://beast.bio.ed.ac.uk}}\\
{\tt Bayes Phylogenies} & {\tt
\url{http://www.evolution.rdg.ac.uk/BayesPhy.html}}\\
{\tt PHASE} & {\tt \url{http://www.bioinf.man.ac.uk/resources/phase/}}\\
{\tt PhyloBayes} & {\tt
\url{http://www.atgc-montpellier.fr/phylobayes/binaries.php}}\\
& (a web server is near there too)\\
{\tt PAML} & {\tt \url{http://abacus.gene.ucl.ac.uk/software/paml.html}} \\
{\tt BAMBE} & {\tt \url{http://www.mathcs.duq.edu/larget/bambe.html}} \\
{\tt p4} & {\tt \url{http://p4.nhm.ac.uk/}}
\end{tabular}
}
}

\noindent
Alternatively, you can explore the other Bayesian programs in the
Bayesian Inference section of the list of programs by Methods in my
``Phylogeny Programs'' web pages (use that phrase to find the pages).
\medskip

\begin{itemize}
\item Make the models as similar as you can to the ones used in the
likelihood analysis.
\item Show a tree summarizing the results of Bayesian inference on your
data, with clade posterior probabilities.
\item Explain what prior on trees (and what priors on parameter values)
were used.  Don't just say ``the default priors'' but describe them.
\item Compare the clade probabilities in the Bayesian analysis to the
bootstrap $P$ values that you found in homework \#3.
\item If there is a parameter in your model for transition/transversion
ratio, have the program give it a prior, and allow it to vary during the
run.  What information did the program give you on the posterior on
the value of this parameter?
\item If you can, do more than one Bayesian run.  Do they get similar
results?  How did you decide how long to run the Bayesian program?
\item If you choose rather different priors, does this change the results?
\item Comment on the biological reasonableness of the results, if possible.
\item Comment on the programs you used, how well they worked and how easy they
were to use.
\item Report this to me in a short (3-5 pages or so) report.  Show some
results (including a tree with clade probabilities).
\end{itemize}
\medskip

\noindent
e-mail me {\tt (joe (at) gs.washington.edu)} with a report in PDF or MS Word
(.DOCX or .DOC) format.
format on the results.

\vfill

\vfill

\end{document}

