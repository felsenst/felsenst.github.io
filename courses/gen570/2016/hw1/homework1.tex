\documentclass[12pt]{article}
\def\baselinestretch{1.0}
\setlength{\topmargin}{0pt}
\setlength{\textheight}{640pt}
\setlength{\oddsidemargin}{0pt}
\setlength{\evensidemargin}{60pt}
\setlength{\textwidth}{477pt}
\setlength{\footskip}{30pt}
\parindent 0.3in
\hyphenpenalty=10000
\tolerance=10000
\pagestyle{empty}

\begin{document}
{\large
\noindent
Genome 570 \hfill Winter, 2016\\
Phylogenetic Inference \hfill J. Felsenstein\\
}

\vfill

\centerline{Homework no. 1}

\centerline{Due Wednesday, January 20}
\bigskip

\begin{itemize}
\item Get a computer program or programs that can do parsimony on nucleotide or protein
sequences, or on discrete characters.  You should also get one that is
able to do
branch-and-bound search as well as heuristic search.  For a list of programs,
many of them free, see\\
\hspace*{0in} \qquad {\tt http://evolution.gs.washington.edu/phylip/software.html}\\
or type ``Phylogeny Programs'' into Google.  Some possibilities include
{\tt PAUP*}, {\tt PHYLIP}, {\tt MEGA}, {\tt TNT}.  Some of the other
excellent general-purpose programs there may not have both heuristic search
and branch-and-bound.   It is OK to get several programs that together have
these capabilities -- they need not all be in one program.
\item Get a data set, either molecular sequences or discrete characters.
It should have at least 20 sequences (or species) and of course these should
be aligned.  More than 20 (say about 40) is better.  For molecular data sets there should be at least 200 sites.
(500 or more is better). If you do not have your own data sets
see the course web site where there both some individual data sets and
some database web sites where aligned sequences can be downloaded.  (Some of
the datasets available there have too few species for this assignment).
\item Run heuristic searches for the most parsimonious trees on data sets of
size 10, 15, 20, ... species (or sequences).
\begin{itemize}
\item  Explain exactly what choice of rearrangement
strategy was used:  
Does the program use nearest-neighbor interchanges?
SPR? TBR? Sequential addition?  Usually the general strategy is explained in
the program documentation.
\end{itemize}
\item How much time do these take?  (For some data file formats
you can do different size data sets just by deleting whole species).  Plot
the run time against number of species.  It may be best to use log-log plot
for that.  Report on whether the run time seems to be proportional to a
power of the number of species, and what that power is.
\item Run branch and bound searches on some of those same data set sizes
(start with the smallest ones and stop when it seems likely to take too long).  Make sure
that you set parameters so that each search (that you don't abandon as
taking too long) finishes completely and is
not broken off short of completion.
\begin{itemize}
\item  How much
time do these take?  Does the run become impossibly slow?  
\item  How do the trees
found by branch-and-bound compare in number and parsimony score with heuristic searches of the
same dataset?
\end{itemize}
\item Are the trees found reasonable?  Why or why not?
\item Report the results to me in a short (2-4 pages or so) report.  Show some
results if needed.  I prefer to receive the report in PDF format, but can
read Microsoft Word (.DOC or .docx) format if needed.
\item Comment on the program you used, how well it worked and how easy it
was to use.
\end{itemize}
\bigskip

You should e-mail me {\tt (joe (at) gs.washington.edu)} with the report on the results.

\vfill

\vfill

\end{document}

