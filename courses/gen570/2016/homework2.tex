\documentclass[12pt]{article}
\def\baselinestretch{1.0}
\setlength{\topmargin}{-60pt}
\setlength{\textheight}{720pt}
\setlength{\oddsidemargin}{0pt}
\setlength{\evensidemargin}{60pt}
\setlength{\textwidth}{477pt}
%\setlength{\footskip}{30pt}
\parindent 0.3in
\hyphenpenalty=10000
\tolerance=10000
\pagestyle{empty}

\begin{document}
{\large
\noindent
Genome 570 \hfill Winter, 2016\\
Phylogenetic Inference \hfill J. Felsenstein\\
}

\centerline{Homework no. 2}

\centerline{Due Friday, February 5 at the end of the evening}
\bigskip

\begin{itemize}
\item Use a data set of sequences of the same gene, over the same set of
species, with DNA data, and also one with protein sequences.  I have provided
27 such data sets available on the course webpage, and you can use those, or
you can download your own from Genbank.  Note that the DNA data sets on the
webpage are a bit different from the ones used for the previous homework, as
some regions of the gene are now omitted as dubious in quality or quality oif
alignment.  So if you use those, download them again.  Make sure to tell me
which gene you used.
\item As before,
the data sets should have about 40 sequences (or species).
\item Use a program (or feature of a program) that computes an appropriate
matrix of distances for the species from DNA sequences. Also use a program or feature that
computes a matrix of distances for the same species from the protein sequences.
Be sure to report what program and what distances you used, and why.
Does the distance you used
cope with possible differences of rate of 
evolution among sites.  (By the way, if there are deletions in the sequences, most distance
programs will ignore those sites for any pair of sequences where one or both
have the gap -- but will {\it not} drop them from the whole data set.)
\item Use a program or programs that compute a trees by a distance matrix method
from these two distance matrices.  These can either use a least squares method
(such as the Fitch-Margoliash method), or the neighbor-joining method.  Or if
you are enthusiastic, both.  Show the trees.
\item Are these trees rooted or unrooted?  If rooted, is that by outgroup?
\item Comment on the speed of the methods (exact timings not needed).
\item In addition, run the UPGMA method for each of the two distance matrices.
Are these trees rooted?  If so, is that by outgroup?
\item How different are the DNA and protein trees in the non-UPGMA method and
in the UPGMA method?
\item How reasonable are these trees?
\item In which parts of the tree are the DNA trees most accurate (recent
divergences or ancient divergences?  In which parts of the tree are the protein
trees most accurate?
\item Comment on how well the programs functioned and how easy or hard they were to use.
\item Report the results to me in a short (2-5 pages or so) report.  Show some
results if needed.
\end{itemize}
\bigskip

There are many distance matrix programs.  Aside PHYLIP, MEGA, and PAUP*, you
will find many others listed in my webpages on ``Phylogeny Programs''.
\bigskip

{\bf Hints for those using PHYLIP programs:}
After using Dnadist or Protdist to compute a distance matrix, you have to
rename {\tt outfile} to {\tt infile} (or else when you run the distance
program you will overwrite your distances and crash).

\end{document}

